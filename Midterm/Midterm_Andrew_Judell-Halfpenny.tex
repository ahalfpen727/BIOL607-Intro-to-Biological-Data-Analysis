\documentclass[]{article}
\usepackage{lmodern}
\usepackage{amssymb,amsmath}
\usepackage{ifxetex,ifluatex}
\usepackage{fixltx2e} % provides \textsubscript
\ifnum 0\ifxetex 1\fi\ifluatex 1\fi=0 % if pdftex
  \usepackage[T1]{fontenc}
  \usepackage[utf8]{inputenc}
\else % if luatex or xelatex
  \ifxetex
    \usepackage{mathspec}
  \else
    \usepackage{fontspec}
  \fi
  \defaultfontfeatures{Ligatures=TeX,Scale=MatchLowercase}
\fi
% use upquote if available, for straight quotes in verbatim environments
\IfFileExists{upquote.sty}{\usepackage{upquote}}{}
% use microtype if available
\IfFileExists{microtype.sty}{%
\usepackage{microtype}
\UseMicrotypeSet[protrusion]{basicmath} % disable protrusion for tt fonts
}{}
\usepackage[margin=1in]{geometry}
\usepackage{hyperref}
\hypersetup{unicode=true,
            pdftitle={BIOL607: Introduction to Biological Data Analysis Midterm Exam 2018},
            pdfauthor={Andrew Judell-Halfpenny},
            pdfborder={0 0 0},
            breaklinks=true}
\urlstyle{same}  % don't use monospace font for urls
\usepackage{color}
\usepackage{fancyvrb}
\newcommand{\VerbBar}{|}
\newcommand{\VERB}{\Verb[commandchars=\\\{\}]}
\DefineVerbatimEnvironment{Highlighting}{Verbatim}{commandchars=\\\{\}}
% Add ',fontsize=\small' for more characters per line
\usepackage{framed}
\definecolor{shadecolor}{RGB}{248,248,248}
\newenvironment{Shaded}{\begin{snugshade}}{\end{snugshade}}
\newcommand{\KeywordTok}[1]{\textcolor[rgb]{0.13,0.29,0.53}{\textbf{#1}}}
\newcommand{\DataTypeTok}[1]{\textcolor[rgb]{0.13,0.29,0.53}{#1}}
\newcommand{\DecValTok}[1]{\textcolor[rgb]{0.00,0.00,0.81}{#1}}
\newcommand{\BaseNTok}[1]{\textcolor[rgb]{0.00,0.00,0.81}{#1}}
\newcommand{\FloatTok}[1]{\textcolor[rgb]{0.00,0.00,0.81}{#1}}
\newcommand{\ConstantTok}[1]{\textcolor[rgb]{0.00,0.00,0.00}{#1}}
\newcommand{\CharTok}[1]{\textcolor[rgb]{0.31,0.60,0.02}{#1}}
\newcommand{\SpecialCharTok}[1]{\textcolor[rgb]{0.00,0.00,0.00}{#1}}
\newcommand{\StringTok}[1]{\textcolor[rgb]{0.31,0.60,0.02}{#1}}
\newcommand{\VerbatimStringTok}[1]{\textcolor[rgb]{0.31,0.60,0.02}{#1}}
\newcommand{\SpecialStringTok}[1]{\textcolor[rgb]{0.31,0.60,0.02}{#1}}
\newcommand{\ImportTok}[1]{#1}
\newcommand{\CommentTok}[1]{\textcolor[rgb]{0.56,0.35,0.01}{\textit{#1}}}
\newcommand{\DocumentationTok}[1]{\textcolor[rgb]{0.56,0.35,0.01}{\textbf{\textit{#1}}}}
\newcommand{\AnnotationTok}[1]{\textcolor[rgb]{0.56,0.35,0.01}{\textbf{\textit{#1}}}}
\newcommand{\CommentVarTok}[1]{\textcolor[rgb]{0.56,0.35,0.01}{\textbf{\textit{#1}}}}
\newcommand{\OtherTok}[1]{\textcolor[rgb]{0.56,0.35,0.01}{#1}}
\newcommand{\FunctionTok}[1]{\textcolor[rgb]{0.00,0.00,0.00}{#1}}
\newcommand{\VariableTok}[1]{\textcolor[rgb]{0.00,0.00,0.00}{#1}}
\newcommand{\ControlFlowTok}[1]{\textcolor[rgb]{0.13,0.29,0.53}{\textbf{#1}}}
\newcommand{\OperatorTok}[1]{\textcolor[rgb]{0.81,0.36,0.00}{\textbf{#1}}}
\newcommand{\BuiltInTok}[1]{#1}
\newcommand{\ExtensionTok}[1]{#1}
\newcommand{\PreprocessorTok}[1]{\textcolor[rgb]{0.56,0.35,0.01}{\textit{#1}}}
\newcommand{\AttributeTok}[1]{\textcolor[rgb]{0.77,0.63,0.00}{#1}}
\newcommand{\RegionMarkerTok}[1]{#1}
\newcommand{\InformationTok}[1]{\textcolor[rgb]{0.56,0.35,0.01}{\textbf{\textit{#1}}}}
\newcommand{\WarningTok}[1]{\textcolor[rgb]{0.56,0.35,0.01}{\textbf{\textit{#1}}}}
\newcommand{\AlertTok}[1]{\textcolor[rgb]{0.94,0.16,0.16}{#1}}
\newcommand{\ErrorTok}[1]{\textcolor[rgb]{0.64,0.00,0.00}{\textbf{#1}}}
\newcommand{\NormalTok}[1]{#1}
\usepackage{graphicx,grffile}
\makeatletter
\def\maxwidth{\ifdim\Gin@nat@width>\linewidth\linewidth\else\Gin@nat@width\fi}
\def\maxheight{\ifdim\Gin@nat@height>\textheight\textheight\else\Gin@nat@height\fi}
\makeatother
% Scale images if necessary, so that they will not overflow the page
% margins by default, and it is still possible to overwrite the defaults
% using explicit options in \includegraphics[width, height, ...]{}
\setkeys{Gin}{width=\maxwidth,height=\maxheight,keepaspectratio}
\IfFileExists{parskip.sty}{%
\usepackage{parskip}
}{% else
\setlength{\parindent}{0pt}
\setlength{\parskip}{6pt plus 2pt minus 1pt}
}
\setlength{\emergencystretch}{3em}  % prevent overfull lines
\providecommand{\tightlist}{%
  \setlength{\itemsep}{0pt}\setlength{\parskip}{0pt}}
\setcounter{secnumdepth}{0}
% Redefines (sub)paragraphs to behave more like sections
\ifx\paragraph\undefined\else
\let\oldparagraph\paragraph
\renewcommand{\paragraph}[1]{\oldparagraph{#1}\mbox{}}
\fi
\ifx\subparagraph\undefined\else
\let\oldsubparagraph\subparagraph
\renewcommand{\subparagraph}[1]{\oldsubparagraph{#1}\mbox{}}
\fi

%%% Use protect on footnotes to avoid problems with footnotes in titles
\let\rmarkdownfootnote\footnote%
\def\footnote{\protect\rmarkdownfootnote}

%%% Change title format to be more compact
\usepackage{titling}

% Create subtitle command for use in maketitle
\newcommand{\subtitle}[1]{
  \posttitle{
    \begin{center}\large#1\end{center}
    }
}

\setlength{\droptitle}{-2em}

  \title{BIOL607: Introduction to Biological Data Analysis Midterm Exam 2018}
    \pretitle{\vspace{\droptitle}\centering\huge}
  \posttitle{\par}
    \author{Andrew Judell-Halfpenny}
    \preauthor{\centering\large\emph}
  \postauthor{\par}
      \predate{\centering\large\emph}
  \postdate{\par}
    \date{Due Nov 9th, 5pm.}


\begin{document}
\maketitle

\begin{Shaded}
\begin{Highlighting}[]
\CommentTok{#knitr::opts_chunk$set(echo = FALSE)}
\KeywordTok{library}\NormalTok{(ggplot2);}\KeywordTok{library}\NormalTok{(tidyverse)}
\end{Highlighting}
\end{Shaded}

\begin{verbatim}
## -- Attaching packages ----------------------------------------------------------- tidyverse 1.2.1 --
\end{verbatim}

\begin{verbatim}
## v tibble  1.4.2     v purrr   0.2.5
## v tidyr   0.8.2     v dplyr   0.7.7
## v readr   1.1.1     v stringr 1.3.1
## v tibble  1.4.2     v forcats 0.3.0
\end{verbatim}

\begin{verbatim}
## -- Conflicts -------------------------------------------------------------- tidyverse_conflicts() --
## x dplyr::filter() masks stats::filter()
## x dplyr::lag()    masks stats::lag()
\end{verbatim}

\begin{Shaded}
\begin{Highlighting}[]
\KeywordTok{library}\NormalTok{(stats4);}\KeywordTok{library}\NormalTok{(MASS);}\KeywordTok{library}\NormalTok{(pwr)}
\end{Highlighting}
\end{Shaded}

\begin{verbatim}
## 
## Attaching package: 'MASS'
\end{verbatim}

\begin{verbatim}
## The following object is masked from 'package:dplyr':
## 
##     select
\end{verbatim}

\begin{Shaded}
\begin{Highlighting}[]
\KeywordTok{library}\NormalTok{(profileModel);}\KeywordTok{library}\NormalTok{(modelr)}
\KeywordTok{library}\NormalTok{(viridis); }\KeywordTok{library}\NormalTok{(bbmle);}\KeywordTok{library}\NormalTok{(broom)}
\end{Highlighting}
\end{Shaded}

\begin{verbatim}
## Loading required package: viridisLite
\end{verbatim}

\begin{verbatim}
## 
## Attaching package: 'bbmle'
\end{verbatim}

\begin{verbatim}
## The following object is masked from 'package:dplyr':
## 
##     slice
\end{verbatim}

\begin{verbatim}
## 
## Attaching package: 'broom'
\end{verbatim}

\begin{verbatim}
## The following object is masked from 'package:modelr':
## 
##     bootstrap
\end{verbatim}

\begin{Shaded}
\begin{Highlighting}[]
\KeywordTok{library}\NormalTok{(pastecs); }\KeywordTok{library}\NormalTok{(Hmisc); }\KeywordTok{library}\NormalTok{(psych)}
\end{Highlighting}
\end{Shaded}

\begin{verbatim}
## 
## Attaching package: 'pastecs'
\end{verbatim}

\begin{verbatim}
## The following objects are masked from 'package:dplyr':
## 
##     first, last
\end{verbatim}

\begin{verbatim}
## The following object is masked from 'package:tidyr':
## 
##     extract
\end{verbatim}

\begin{verbatim}
## Loading required package: lattice
\end{verbatim}

\begin{verbatim}
## Loading required package: survival
\end{verbatim}

\begin{verbatim}
## Loading required package: Formula
\end{verbatim}

\begin{verbatim}
## 
## Attaching package: 'Hmisc'
\end{verbatim}

\begin{verbatim}
## The following objects are masked from 'package:dplyr':
## 
##     src, summarize
\end{verbatim}

\begin{verbatim}
## The following objects are masked from 'package:base':
## 
##     format.pval, units
\end{verbatim}

\begin{verbatim}
## 
## Attaching package: 'psych'
\end{verbatim}

\begin{verbatim}
## The following object is masked from 'package:Hmisc':
## 
##     describe
\end{verbatim}

\begin{verbatim}
## The following object is masked from 'package:modelr':
## 
##     heights
\end{verbatim}

\begin{verbatim}
## The following objects are masked from 'package:ggplot2':
## 
##     %+%, alpha
\end{verbatim}

\begin{Shaded}
\begin{Highlighting}[]
\KeywordTok{library}\NormalTok{(brms);}\KeywordTok{library}\NormalTok{(MCMCpack); }\KeywordTok{library}\NormalTok{(bayesplot)}
\end{Highlighting}
\end{Shaded}

\begin{verbatim}
## Loading required package: Rcpp
\end{verbatim}

\begin{verbatim}
## Loading 'brms' package (version 2.6.0). Useful instructions
## can be found by typing help('brms'). A more detailed introduction
## to the package is available through vignette('brms_overview').
## Run theme_set(theme_default()) to use the default bayesplot theme.
\end{verbatim}

\begin{verbatim}
## 
## Attaching package: 'brms'
\end{verbatim}

\begin{verbatim}
## The following object is masked from 'package:psych':
## 
##     cs
\end{verbatim}

\begin{verbatim}
## The following object is masked from 'package:survival':
## 
##     kidney
\end{verbatim}

\begin{verbatim}
## The following object is masked from 'package:bbmle':
## 
##     parnames
\end{verbatim}

\begin{verbatim}
## Loading required package: coda
\end{verbatim}

\begin{verbatim}
## ##
## ## Markov Chain Monte Carlo Package (MCMCpack)
\end{verbatim}

\begin{verbatim}
## ## Copyright (C) 2003-2018 Andrew D. Martin, Kevin M. Quinn, and Jong Hee Park
\end{verbatim}

\begin{verbatim}
## ##
## ## Support provided by the U.S. National Science Foundation
\end{verbatim}

\begin{verbatim}
## ## (Grants SES-0350646 and SES-0350613)
## ##
\end{verbatim}

\begin{verbatim}
## This is bayesplot version 1.6.0
\end{verbatim}

\begin{verbatim}
## - Online documentation and vignettes at mc-stan.org/bayesplot
\end{verbatim}

\begin{verbatim}
## - bayesplot theme set to bayesplot::theme_default()
\end{verbatim}

\begin{verbatim}
##    * Does _not_ affect other ggplot2 plots
\end{verbatim}

\begin{verbatim}
##    * See ?bayesplot_theme_set for details on theme setting
\end{verbatim}

\begin{Shaded}
\begin{Highlighting}[]
\KeywordTok{library}\NormalTok{(lme4);}\KeywordTok{library}\NormalTok{(brms);}\KeywordTok{library}\NormalTok{(rstan)}
\end{Highlighting}
\end{Shaded}

\begin{verbatim}
## Loading required package: Matrix
\end{verbatim}

\begin{verbatim}
## 
## Attaching package: 'Matrix'
\end{verbatim}

\begin{verbatim}
## The following object is masked from 'package:tidyr':
## 
##     expand
\end{verbatim}

\begin{verbatim}
## 
## Attaching package: 'lme4'
\end{verbatim}

\begin{verbatim}
## The following object is masked from 'package:brms':
## 
##     ngrps
\end{verbatim}

\begin{verbatim}
## Loading required package: StanHeaders
\end{verbatim}

\begin{verbatim}
## rstan (Version 2.18.2, GitRev: 2e1f913d3ca3)
\end{verbatim}

\begin{verbatim}
## For execution on a local, multicore CPU with excess RAM we recommend calling
## options(mc.cores = parallel::detectCores()).
## To avoid recompilation of unchanged Stan programs, we recommend calling
## rstan_options(auto_write = TRUE)
\end{verbatim}

\begin{verbatim}
## 
## Attaching package: 'rstan'
\end{verbatim}

\begin{verbatim}
## The following object is masked from 'package:coda':
## 
##     traceplot
\end{verbatim}

\begin{verbatim}
## The following object is masked from 'package:psych':
## 
##     lookup
\end{verbatim}

\begin{verbatim}
## The following object is masked from 'package:pastecs':
## 
##     extract
\end{verbatim}

\begin{verbatim}
## The following object is masked from 'package:tidyr':
## 
##     extract
\end{verbatim}

\begin{Shaded}
\begin{Highlighting}[]
\KeywordTok{rstan_options}\NormalTok{(}\DataTypeTok{auto_write=}\OtherTok{TRUE}\NormalTok{)}
\KeywordTok{options}\NormalTok{(}\DataTypeTok{mc.cores=}\NormalTok{parallel}\OperatorTok{::}\KeywordTok{detectCores}\NormalTok{())}
\NormalTok{seal.csv<-}\KeywordTok{read_csv}\NormalTok{(}\StringTok{"http://www.zoology.ubc.ca/~whitlock/ABD/teaching/datasets/17/17e8ShrinkingSeals%20Trites%201996.csv"}\NormalTok{)}
\end{Highlighting}
\end{Shaded}

\begin{verbatim}
## Parsed with column specification:
## cols(
##   age.days = col_integer(),
##   length.cm = col_integer()
## )
\end{verbatim}

\begin{Shaded}
\begin{Highlighting}[]
\NormalTok{quaildata<-}\KeywordTok{read_csv}\NormalTok{(}\StringTok{"https://datadryad.org/bitstream/handle/10255/dryad.51265/Morphology%20data.csv?sequence=1"}\NormalTok{)}
\end{Highlighting}
\end{Shaded}

\begin{verbatim}
## Parsed with column specification:
## cols(
##   `Bird #` = col_integer(),
##   Sex = col_character(),
##   `Age (days)` = col_integer(),
##   `Exp. Temp. (degree C)` = col_integer(),
##   `Mass (g)` = col_double(),
##   `Tarsus (mm)` = col_double(),
##   `Culmen (mm)` = col_double(),
##   `Depth (mm)` = col_double(),
##   `Width (mm)` = col_double(),
##   NOTES = col_character()
## )
\end{verbatim}

\begin{Shaded}
\begin{Highlighting}[]
\NormalTok{s.mean=}\FloatTok{3730.246}\NormalTok{; s.sd=}\FloatTok{1293.485}\NormalTok{;s.ages.range =}\KeywordTok{c}\NormalTok{(}\DecValTok{958}\OperatorTok{:}\DecValTok{8353}\NormalTok{)}
\NormalTok{s.slope=}\FloatTok{0.00237}\NormalTok{;s.intercept=}\FloatTok{115.767}\NormalTok{;s.sigma =}\StringTok{ }\FloatTok{5.6805}
\end{Highlighting}
\end{Shaded}

\section{1) Sampling your system (10
points)}\label{sampling-your-system-10-points}

\begin{quote}
Each of you has a study system your work in and a question of interest.
Give an example of one variable that you would sample in order to get a
sense of its variation in nature. Describe, in detail, how you would
sample for the population of that variable in order to understand its
distribution. Questions to consider include, but are not limited to:
Just what is your sample versus your population? What would your
sampling design be? Why would you design it that particular way? What
are potential confounding influences of both sampling technique and
sample design that you need to be careful to avoid? What statistical
distribution might the variable take, and why?
\end{quote}

\section{1. Answer:}\label{answer}

\begin{quote}
I am interested in sampling the population to identify the features of
the human genome that are pervasively transcribed/consituitively
expressed. In order to identify theses features, public repositories
(GEO-profiles) could be queryied for RNA-Seq data. All available human
RNA-Seq data would be culled and mapped to the same genome
(hg38/GRCh38). Gene expression would be quantified in fragments per
kilobase of transcript per million mapped read (FPKM). For each
comparative experiment, correlation coeficients would be calculated for
each gene across all conditions. Only genes with (arbitrarily) high
correlation (90\%) across both conditions would be added to a librabry
of pontentially constitiutively expressed genes for that experiment. For
each comparative experiment, correlation coefficients would be caclcuted
for each gene to identify consistently expressed features independent of
condition. The initial library of constituitively expressed genes could
be culled from hierarchical clustering of the correlation coefficients
of gene expression across all conditions for several experiments. This
could also be done with a principal component analysis (PCA) mutli
dimensional scaling (MDS) analysis of the gene expression across all
samples from many experiments. Rather than looking at the direction of
maximal variation, the loadings for the features of minimal variation
could be used to identify genes with consistenet expression across all
conditions and all experiments. Genes from new experiments could then be
added to the library of constituitively expressed features by performing
ANOVA hypothesis tests. The consitituitively expressed features and the
distribution of their expression could then be queried to produce a
canonical control group for RNA-Seq analyses.
\end{quote}

\begin{center}\rule{0.5\linewidth}{\linethickness}\end{center}

\section{2) Let's get philosophical. (10
points)}\label{lets-get-philosophical.-10-points}

\begin{quote}
Are you a frequentist, likelihoodist, or Bayesian? Why? Include in your
answer why you prefer the inferential tools (e.g.~confidence intervals,
test statistics, posterior probabilities, etc.) of your chosen worldview
and why you do not like the ones of the other one. This includes
defining just what those different tools mean! extra credit for citing
and discussing outside sources - one point per source/point \# 2.
Answer: Of the three current statistical philosophies, I prefer
likelihoodist. The likelihood ratio is a simple tool to identify model
fit. The metric for significance is the deviance, a modified version of
the test statistic. The philosophy is an derivative of frequentist
statistics but lacks the obtuse frequentist language. Likelihood
consideration of how well data support a given hypothesis provides more
utility than the frequentist quest for extreme values. Frequentist
statistics is more familiar to me but the idea of failing to reject a
null hypothesis is a ridiculous endeavour in mental aeorobics.The
frequentist p-value and condfidence interval are the most familiar
statistical tools available. Bayesian statistics is a new and esoteric
concept for me despite its utility. Bayesian statistics incorporates a
prior belief into the likelihoodist data generating model and normalizes
by the marginal probability of all outcomes.
\end{quote}

\begin{center}\rule{0.5\linewidth}{\linethickness}\end{center}

\section{3) Power (20 points)}\label{power-20-points}

\begin{quote}
We have a lot of aspects of the sample of data that we collect which can
alter the power of our linear regressions. Slope, Intercept, Reidual
variance, Sample Size, Range of X values Choose three of the above
properties and demonstrate how they alter power of an F-test from a
linear regression using at least three different alpha levels (more if
you want!) As a baseline of the parameters, ( Your call what
distribution to use for seal age simulation) slope = 0.00237,
intercept=115.767, sigma = 5.6805, range of seal ages = 958 to 8353, or,
if you prefer, seal ages ∼ N(3730.246, 1293.485).
\end{quote}

\begin{Shaded}
\begin{Highlighting}[]
\NormalTok{knitr}\OperatorTok{::}\NormalTok{opts_chunk}\OperatorTok{$}\KeywordTok{set}\NormalTok{(}\DataTypeTok{echo =} \OtherTok{TRUE}\NormalTok{)}

\NormalTok{s.mean=}\FloatTok{3730.246}\NormalTok{; s.sd=}\FloatTok{1293.485}\NormalTok{;s.ages.range =}\KeywordTok{c}\NormalTok{(}\DecValTok{958}\NormalTok{,}\DecValTok{8353}\NormalTok{)}
\NormalTok{s.slope=}\FloatTok{0.00237}\NormalTok{;s.intercept=}\FloatTok{115.767}\NormalTok{;s.sigma =}\StringTok{ }\FloatTok{5.6805}

\NormalTok{seal.lm<-}\KeywordTok{lm}\NormalTok{(length.cm }\OperatorTok{~}\StringTok{ }\NormalTok{age.days, }\DataTypeTok{data=}\NormalTok{seal.csv)}
\KeywordTok{summary}\NormalTok{(seal.lm)}
\end{Highlighting}
\end{Shaded}

\begin{verbatim}
## 
## Call:
## lm(formula = length.cm ~ age.days, data = seal.csv)
## 
## Residuals:
##      Min       1Q   Median       3Q      Max 
## -19.8700  -3.8279   0.0304   3.7541  21.6874 
## 
## Coefficients:
##              Estimate Std. Error t value Pr(>|t|)    
## (Intercept) 1.158e+02  1.764e-01  656.37   <2e-16 ***
## age.days    2.371e-03  4.467e-05   53.06   <2e-16 ***
## ---
## Signif. codes:  0 '***' 0.001 '**' 0.01 '*' 0.05 '.' 0.1 ' ' 1
## 
## Residual standard error: 5.681 on 9663 degrees of freedom
## Multiple R-squared:  0.2256, Adjusted R-squared:  0.2256 
## F-statistic:  2816 on 1 and 9663 DF,  p-value: < 2.2e-16
\end{verbatim}

\begin{Shaded}
\begin{Highlighting}[]
\KeywordTok{anova}\NormalTok{(seal.lm)}
\end{Highlighting}
\end{Shaded}

\begin{verbatim}
## Analysis of Variance Table
## 
## Response: length.cm
##             Df Sum Sq Mean Sq F value    Pr(>F)    
## age.days     1  90862   90862  2815.8 < 2.2e-16 ***
## Residuals 9663 311807      32                      
## ---
## Signif. codes:  0 '***' 0.001 '**' 0.01 '*' 0.05 '.' 0.1 ' ' 1
\end{verbatim}

\begin{Shaded}
\begin{Highlighting}[]
\NormalTok{seal.sim <-}\StringTok{ }\NormalTok{seal.csv }\OperatorTok
\StringTok{   }\KeywordTok{filter}\NormalTok{(age.days }\OperatorTok{>=}\StringTok{ }\NormalTok{s.ages.range[}\DecValTok{1}\NormalTok{], age.days }\OperatorTok{<=}\StringTok{ }\NormalTok{s.ages.range[}\DecValTok{2}\NormalTok{]) }\OperatorTok
\StringTok{   }\KeywordTok{mutate}\NormalTok{(}\DataTypeTok{samp_size=}\DecValTok{1}\OperatorTok{:}\KeywordTok{nrow}\NormalTok{(seal.csv)) }\OperatorTok
\StringTok{   }\KeywordTok{mutate}\NormalTok{(}\DataTypeTok{se_y =} \KeywordTok{sd}\NormalTok{(age.days)}\OperatorTok{/}\StringTok{ }\KeywordTok{sqrt}\NormalTok{(samp_size)) }\OperatorTok
\StringTok{   }\KeywordTok{group_by}\NormalTok{(samp_size) }\OperatorTok
\StringTok{   }\KeywordTok{mutate}\NormalTok{(}\DataTypeTok{z =}\NormalTok{ (age.days}\OperatorTok{-}\NormalTok{s.mean)}\OperatorTok{/}\NormalTok{se_y) }\OperatorTok
\StringTok{   }\KeywordTok{mutate}\NormalTok{(}\DataTypeTok{slope =} \KeywordTok{c}\NormalTok{(length.cm}\OperatorTok{/}\NormalTok{age.days)) }\OperatorTok
\StringTok{   }\KeywordTok{mutate}\NormalTok{(}\DataTypeTok{p =} \DecValTok{2}\OperatorTok{*}\KeywordTok{pnorm}\NormalTok{(}\KeywordTok{abs}\NormalTok{(z), }\DataTypeTok{lower.tail=}\OtherTok{FALSE}\NormalTok{)) }\OperatorTok
\StringTok{   }\KeywordTok{add_residuals}\NormalTok{(seal.lm) }\OperatorTok
\StringTok{   }\KeywordTok{add_predictions}\NormalTok{(seal.lm) }\OperatorTok
\StringTok{   }\KeywordTok{ungroup}\NormalTok{()}

\CommentTok{# Assumptions, lack of correlation}
\KeywordTok{qplot}\NormalTok{(pred, resid, }\DataTypeTok{data=}\NormalTok{seal.sim) }\OperatorTok{+}
\StringTok{  }\KeywordTok{stat_smooth}\NormalTok{(}\DataTypeTok{method=}\StringTok{"lm"}\NormalTok{)}
\end{Highlighting}
\end{Shaded}

\includegraphics{Midterm_Andrew_Judell-Halfpenny_files/figure-latex/power-data-prep-1.pdf}

\section{Sample Size Influence on Power of
F-Test}\label{sample-size-influence-on-power-of-f-test}

\begin{quote}
The following graphs depict power as a function of sample size at
several \[\alpha levels (\]\alpha=0.01 ,
\[\alpha=0.05 , and \]\alpha=0.1 ). Increasing \$\$\alpha essentially
increases power for similar sample sizes
\end{quote}

\begin{Shaded}
\begin{Highlighting}[]
\NormalTok{knitr}\OperatorTok{::}\NormalTok{opts_chunk}\OperatorTok{$}\KeywordTok{set}\NormalTok{(}\DataTypeTok{echo =} \OtherTok{TRUE}\NormalTok{)}

\NormalTok{sample_sizes=}\DecValTok{2}\OperatorTok{:}\KeywordTok{nrow}\NormalTok{(seal.csv)}
\NormalTok{power_f_test=}\KeywordTok{pwr.f2.test}\NormalTok{(}\DataTypeTok{u=}\NormalTok{sample_sizes,}\DataTypeTok{v=}\NormalTok{sample_sizes, }\DataTypeTok{f2 =} \FloatTok{0.1}\NormalTok{ , }\DataTypeTok{sig.level =} \FloatTok{0.01}\NormalTok{, }\DataTypeTok{power =}\NormalTok{ )}
\KeywordTok{plot}\NormalTok{(sample_sizes,power_f_test}\OperatorTok{$}\NormalTok{power,}\DataTypeTok{col=}\StringTok{"blue"}\NormalTok{,}
     \DataTypeTok{ylab=}\StringTok{'statistical power'}\NormalTok{,}\DataTypeTok{xlab=}\StringTok{'sample size'}\NormalTok{, }\DataTypeTok{ylim =} \KeywordTok{c}\NormalTok{(}\DecValTok{0}\NormalTok{,}\DecValTok{1}\NormalTok{),}\DataTypeTok{main=}\StringTok{"Sample size influence on power at alpha=0.01"}\NormalTok{)}
\KeywordTok{abline}\NormalTok{(}\DataTypeTok{h =} \FloatTok{0.80}\NormalTok{, }\DataTypeTok{lty=}\DecValTok{2}\NormalTok{, }\DataTypeTok{lwd=}\DecValTok{2}\NormalTok{,}\DataTypeTok{col=}\StringTok{'red'}\NormalTok{)}
\KeywordTok{abline}\NormalTok{( }\DataTypeTok{v =} \DecValTok{1200}\NormalTok{, }\DataTypeTok{col =} \StringTok{"red"}\NormalTok{, }\DataTypeTok{lty =} \DecValTok{3}\NormalTok{)}
\KeywordTok{text}\NormalTok{(}\DecValTok{1}\NormalTok{,}\DecValTok{0}\NormalTok{, }\StringTok{"abline( v = 1200 )"}\NormalTok{, }\DataTypeTok{col =} \StringTok{"gray60"}\NormalTok{, }\DataTypeTok{adj =} \KeywordTok{c}\NormalTok{(}\DecValTok{0}\NormalTok{, }\FloatTok{-.1}\NormalTok{))}
\end{Highlighting}
\end{Shaded}

\includegraphics{Midterm_Andrew_Judell-Halfpenny_files/figure-latex/power-sample-size-1.pdf}

\begin{Shaded}
\begin{Highlighting}[]
\NormalTok{power_f2_test=}\KeywordTok{pwr.f2.test}\NormalTok{(}\DataTypeTok{u=}\NormalTok{sample_sizes,}\DataTypeTok{v=}\NormalTok{sample_sizes, }\DataTypeTok{f2 =} \FloatTok{0.1}\NormalTok{ , }\DataTypeTok{sig.level =} \FloatTok{0.05}\NormalTok{, }\DataTypeTok{power =}\NormalTok{ )}
\KeywordTok{plot}\NormalTok{(sample_sizes,power_f2_test}\OperatorTok{$}\NormalTok{power,}\DataTypeTok{col=}\StringTok{"blue"}\NormalTok{,}
     \DataTypeTok{ylab=}\StringTok{'statistical power'}\NormalTok{,}\DataTypeTok{xlab=}\StringTok{'sample size'}\NormalTok{, }\DataTypeTok{ylim =} \KeywordTok{c}\NormalTok{(}\DecValTok{0}\NormalTok{,}\DecValTok{1}\NormalTok{),}\DataTypeTok{main=}\StringTok{"Sample size influence on power at alpha=0.05"}\NormalTok{)}
\KeywordTok{abline}\NormalTok{(}\DataTypeTok{h =} \FloatTok{0.80}\NormalTok{, }\DataTypeTok{lty=}\DecValTok{2}\NormalTok{, }\DataTypeTok{lwd=}\DecValTok{2}\NormalTok{,}\DataTypeTok{col=}\StringTok{'red'}\NormalTok{)}
\KeywordTok{abline}\NormalTok{( }\DataTypeTok{v =} \DecValTok{1200}\NormalTok{, }\DataTypeTok{col =} \StringTok{"red"}\NormalTok{, }\DataTypeTok{lty =} \DecValTok{3}\NormalTok{)}
\KeywordTok{text}\NormalTok{(}\DecValTok{1}\NormalTok{,}\DecValTok{0}\NormalTok{, }\StringTok{"abline( v = 1200 )"}\NormalTok{, }\DataTypeTok{col =} \StringTok{"gray60"}\NormalTok{, }\DataTypeTok{adj =} \KeywordTok{c}\NormalTok{(}\DecValTok{0}\NormalTok{, }\FloatTok{-.1}\NormalTok{))}
\end{Highlighting}
\end{Shaded}

\includegraphics{Midterm_Andrew_Judell-Halfpenny_files/figure-latex/power-sample-size-2.pdf}

\begin{Shaded}
\begin{Highlighting}[]
\NormalTok{power_f2_test=}\KeywordTok{pwr.f2.test}\NormalTok{(}\DataTypeTok{u=}\NormalTok{sample_sizes,}\DataTypeTok{v=}\NormalTok{sample_sizes, }\DataTypeTok{f2 =} \FloatTok{0.1}\NormalTok{ , }\DataTypeTok{sig.level =} \FloatTok{0.1}\NormalTok{, }\DataTypeTok{power =}\NormalTok{ )}
\KeywordTok{plot}\NormalTok{(sample_sizes,power_f2_test}\OperatorTok{$}\NormalTok{power,}\DataTypeTok{col=}\StringTok{"blue"}\NormalTok{,}
     \DataTypeTok{ylab=}\StringTok{'statistical power'}\NormalTok{,}\DataTypeTok{xlab=}\StringTok{'sample size'}\NormalTok{, }\DataTypeTok{ylim =} \KeywordTok{c}\NormalTok{(}\DecValTok{0}\NormalTok{,}\DecValTok{1}\NormalTok{),}\DataTypeTok{main=}\StringTok{"Sample size influence on power at alpha=0.1"}\NormalTok{)}
\KeywordTok{abline}\NormalTok{(}\DataTypeTok{h =} \FloatTok{0.80}\NormalTok{, }\DataTypeTok{lty=}\DecValTok{2}\NormalTok{, }\DataTypeTok{lwd=}\DecValTok{2}\NormalTok{,}\DataTypeTok{col=}\StringTok{'red'}\NormalTok{)}
\KeywordTok{abline}\NormalTok{( }\DataTypeTok{v =} \DecValTok{1200}\NormalTok{, }\DataTypeTok{col =} \StringTok{"red"}\NormalTok{, }\DataTypeTok{lty =} \DecValTok{3}\NormalTok{)}
\KeywordTok{text}\NormalTok{(}\DecValTok{1}\NormalTok{,}\DecValTok{0}\NormalTok{, }\StringTok{"abline( v = 1200 )"}\NormalTok{, }\DataTypeTok{col =} \StringTok{"gray60"}\NormalTok{, }\DataTypeTok{adj =} \KeywordTok{c}\NormalTok{(}\DecValTok{0}\NormalTok{, }\FloatTok{-.1}\NormalTok{))}
\end{Highlighting}
\end{Shaded}

\includegraphics{Midterm_Andrew_Judell-Halfpenny_files/figure-latex/power-sample-size-3.pdf}

\section{Range of X-values Influence on Power of
F-Test}\label{range-of-x-values-influence-on-power-of-f-test}

\begin{Shaded}
\begin{Highlighting}[]
\NormalTok{knitr}\OperatorTok{::}\NormalTok{opts_chunk}\OperatorTok{$}\KeywordTok{set}\NormalTok{(}\DataTypeTok{echo =} \OtherTok{TRUE}\NormalTok{)}

\NormalTok{seal.sim.xr.}\FloatTok{1.}\NormalTok{lm<-}\KeywordTok{lm}\NormalTok{(length.cm }\OperatorTok{~}\StringTok{ }\NormalTok{age.days, }\DataTypeTok{data=}\NormalTok{seal.sim)}
\KeywordTok{summary}\NormalTok{(seal.sim.xr.}\FloatTok{1.}\NormalTok{lm)}
\end{Highlighting}
\end{Shaded}

\begin{verbatim}
## 
## Call:
## lm(formula = length.cm ~ age.days, data = seal.sim)
## 
## Residuals:
##      Min       1Q   Median       3Q      Max 
## -19.8700  -3.8279   0.0304   3.7541  21.6874 
## 
## Coefficients:
##              Estimate Std. Error t value Pr(>|t|)    
## (Intercept) 1.158e+02  1.764e-01  656.37   <2e-16 ***
## age.days    2.371e-03  4.467e-05   53.06   <2e-16 ***
## ---
## Signif. codes:  0 '***' 0.001 '**' 0.01 '*' 0.05 '.' 0.1 ' ' 1
## 
## Residual standard error: 5.681 on 9663 degrees of freedom
## Multiple R-squared:  0.2256, Adjusted R-squared:  0.2256 
## F-statistic:  2816 on 1 and 9663 DF,  p-value: < 2.2e-16
\end{verbatim}

\begin{Shaded}
\begin{Highlighting}[]
\KeywordTok{anova}\NormalTok{(seal.sim.xr.}\FloatTok{1.}\NormalTok{lm)}
\end{Highlighting}
\end{Shaded}

\begin{verbatim}
## Analysis of Variance Table
## 
## Response: length.cm
##             Df Sum Sq Mean Sq F value    Pr(>F)    
## age.days     1  90862   90862  2815.8 < 2.2e-16 ***
## Residuals 9663 311807      32                      
## ---
## Signif. codes:  0 '***' 0.001 '**' 0.01 '*' 0.05 '.' 0.1 ' ' 1
\end{verbatim}

\begin{Shaded}
\begin{Highlighting}[]
\KeywordTok{var}\NormalTok{(seal.sim}\OperatorTok{$}\NormalTok{age.days)}
\end{Highlighting}
\end{Shaded}

\begin{verbatim}
## [1] 1673103
\end{verbatim}

\begin{Shaded}
\begin{Highlighting}[]
\NormalTok{seal.sim }\OperatorTok
\StringTok{    }\KeywordTok{summarise}\NormalTok{(}\DataTypeTok{power =} \DecValTok{1}\OperatorTok{-}\KeywordTok{sum}\NormalTok{(p}\OperatorTok{>}\FloatTok{0.01}\NormalTok{)}\OperatorTok{/}\KeywordTok{n}\NormalTok{()) }\OperatorTok\StringTok{ }
\StringTok{    }\KeywordTok{ungroup}\NormalTok{()}
\end{Highlighting}
\end{Shaded}

\begin{verbatim}
## # A tibble: 1 x 1
##   power
##   <dbl>
## 1 0.978
\end{verbatim}

\begin{Shaded}
\begin{Highlighting}[]
\NormalTok{seal.sim }\OperatorTok
\StringTok{   }\KeywordTok{summarise}\NormalTok{(}\DataTypeTok{power =} \DecValTok{1}\OperatorTok{-}\KeywordTok{sum}\NormalTok{(p}\OperatorTok{>}\FloatTok{0.05}\NormalTok{)}\OperatorTok{/}\KeywordTok{n}\NormalTok{()) }\OperatorTok\StringTok{ }
\StringTok{   }\KeywordTok{ungroup}\NormalTok{()}
\end{Highlighting}
\end{Shaded}

\begin{verbatim}
## # A tibble: 1 x 1
##   power
##   <dbl>
## 1 0.986
\end{verbatim}

\begin{Shaded}
\begin{Highlighting}[]
\NormalTok{seal.sim }\OperatorTok
\StringTok{   }\KeywordTok{summarise}\NormalTok{(}\DataTypeTok{power =} \DecValTok{1}\OperatorTok{-}\KeywordTok{sum}\NormalTok{(p}\OperatorTok{>}\FloatTok{0.1}\NormalTok{)}\OperatorTok{/}\KeywordTok{n}\NormalTok{()) }\OperatorTok\StringTok{ }
\StringTok{   }\KeywordTok{ungroup}\NormalTok{()}
\end{Highlighting}
\end{Shaded}

\begin{verbatim}
## # A tibble: 1 x 1
##   power
##   <dbl>
## 1 0.990
\end{verbatim}

\begin{Shaded}
\begin{Highlighting}[]
\NormalTok{seal.sim <-}\KeywordTok{subset}\NormalTok{(seal.sim, seal.sim}\OperatorTok{$}\NormalTok{age.days }\OperatorTok{>=}\StringTok{ }\DecValTok{2000} \OperatorTok{&}\StringTok{ }\NormalTok{seal.sim}\OperatorTok{$}\NormalTok{age.days }\OperatorTok{<=}\StringTok{ }\DecValTok{6000}\NormalTok{)}
\KeywordTok{var}\NormalTok{(seal.sim}\OperatorTok{$}\NormalTok{age.days)}
\end{Highlighting}
\end{Shaded}

\begin{verbatim}
## [1] 1147934
\end{verbatim}

\begin{Shaded}
\begin{Highlighting}[]
\NormalTok{seal.sim }\OperatorTok
\StringTok{    }\KeywordTok{summarise}\NormalTok{(}\DataTypeTok{power =} \DecValTok{1}\OperatorTok{-}\KeywordTok{sum}\NormalTok{(p}\OperatorTok{>}\FloatTok{0.01}\NormalTok{)}\OperatorTok{/}\KeywordTok{n}\NormalTok{()) }\OperatorTok\StringTok{ }
\StringTok{    }\KeywordTok{ungroup}\NormalTok{()}
\end{Highlighting}
\end{Shaded}

\begin{verbatim}
## # A tibble: 1 x 1
##   power
##   <dbl>
## 1 0.976
\end{verbatim}

\begin{Shaded}
\begin{Highlighting}[]
\NormalTok{seal.sim }\OperatorTok
\StringTok{   }\KeywordTok{summarise}\NormalTok{(}\DataTypeTok{power =} \DecValTok{1}\OperatorTok{-}\KeywordTok{sum}\NormalTok{(p}\OperatorTok{>}\FloatTok{0.05}\NormalTok{)}\OperatorTok{/}\KeywordTok{n}\NormalTok{()) }\OperatorTok\StringTok{ }
\StringTok{   }\KeywordTok{ungroup}\NormalTok{()}
\end{Highlighting}
\end{Shaded}

\begin{verbatim}
## # A tibble: 1 x 1
##   power
##   <dbl>
## 1 0.984
\end{verbatim}

\begin{Shaded}
\begin{Highlighting}[]
\NormalTok{seal.sim }\OperatorTok
\StringTok{   }\KeywordTok{summarise}\NormalTok{(}\DataTypeTok{power =} \DecValTok{1}\OperatorTok{-}\KeywordTok{sum}\NormalTok{(p}\OperatorTok{>}\FloatTok{0.1}\NormalTok{)}\OperatorTok{/}\KeywordTok{n}\NormalTok{()) }\OperatorTok\StringTok{ }
\StringTok{   }\KeywordTok{ungroup}\NormalTok{()}
\end{Highlighting}
\end{Shaded}

\begin{verbatim}
## # A tibble: 1 x 1
##   power
##   <dbl>
## 1 0.989
\end{verbatim}

\begin{Shaded}
\begin{Highlighting}[]
\NormalTok{seal.sim <-}\KeywordTok{subset}\NormalTok{(seal.sim, seal.sim}\OperatorTok{$}\NormalTok{age.days }\OperatorTok{>=}\StringTok{ }\DecValTok{2500} \OperatorTok{&}\StringTok{ }\NormalTok{seal.sim}\OperatorTok{$}\NormalTok{age.days }\OperatorTok{<=}\StringTok{ }\DecValTok{5500}\NormalTok{)}
\KeywordTok{var}\NormalTok{(seal.sim}\OperatorTok{$}\NormalTok{age.days)}
\end{Highlighting}
\end{Shaded}

\begin{verbatim}
## [1] 704632.6
\end{verbatim}

\begin{Shaded}
\begin{Highlighting}[]
\NormalTok{seal.sim }\OperatorTok
\StringTok{    }\KeywordTok{summarise}\NormalTok{(}\DataTypeTok{power =} \DecValTok{1}\OperatorTok{-}\KeywordTok{sum}\NormalTok{(p}\OperatorTok{>}\FloatTok{0.01}\NormalTok{)}\OperatorTok{/}\KeywordTok{n}\NormalTok{()) }\OperatorTok\StringTok{ }
\StringTok{    }\KeywordTok{ungroup}\NormalTok{()}
\end{Highlighting}
\end{Shaded}

\begin{verbatim}
## # A tibble: 1 x 1
##   power
##   <dbl>
## 1 0.969
\end{verbatim}

\begin{Shaded}
\begin{Highlighting}[]
\NormalTok{seal.sim }\OperatorTok
\StringTok{   }\KeywordTok{summarise}\NormalTok{(}\DataTypeTok{power =} \DecValTok{1}\OperatorTok{-}\KeywordTok{sum}\NormalTok{(p}\OperatorTok{>}\FloatTok{0.05}\NormalTok{)}\OperatorTok{/}\KeywordTok{n}\NormalTok{()) }\OperatorTok\StringTok{ }
\StringTok{   }\KeywordTok{ungroup}\NormalTok{()}
\end{Highlighting}
\end{Shaded}

\begin{verbatim}
## # A tibble: 1 x 1
##   power
##   <dbl>
## 1 0.980
\end{verbatim}

\begin{Shaded}
\begin{Highlighting}[]
\NormalTok{seal.sim }\OperatorTok
\StringTok{   }\KeywordTok{summarise}\NormalTok{(}\DataTypeTok{power =} \DecValTok{1}\OperatorTok{-}\KeywordTok{sum}\NormalTok{(p}\OperatorTok{>}\FloatTok{0.1}\NormalTok{)}\OperatorTok{/}\KeywordTok{n}\NormalTok{()) }\OperatorTok\StringTok{ }
\StringTok{   }\KeywordTok{ungroup}\NormalTok{()}
\end{Highlighting}
\end{Shaded}

\begin{verbatim}
## # A tibble: 1 x 1
##   power
##   <dbl>
## 1 0.986
\end{verbatim}

\begin{quote}
Minimizing the range of X-values in a way that also reduced that
variance of the X-values still lessened the Power of an F-Test in a
similar fashion to a reduction in sample size
\end{quote}

\section{Influence of Slope on Power of
F-Test}\label{influence-of-slope-on-power-of-f-test}

\begin{Shaded}
\begin{Highlighting}[]
\NormalTok{knitr}\OperatorTok{::}\NormalTok{opts_chunk}\OperatorTok{$}\KeywordTok{set}\NormalTok{(}\DataTypeTok{echo =} \OtherTok{TRUE}\NormalTok{)}

\NormalTok{seal.sim <-}\StringTok{ }\NormalTok{seal.csv }\OperatorTok
\StringTok{   }\KeywordTok{filter}\NormalTok{(age.days }\OperatorTok{>=}\StringTok{ }\NormalTok{s.ages.range[}\DecValTok{1}\NormalTok{], age.days }\OperatorTok{<=}\StringTok{ }\NormalTok{s.ages.range[}\DecValTok{2}\NormalTok{]) }\OperatorTok
\StringTok{   }\KeywordTok{mutate}\NormalTok{(}\DataTypeTok{samp_size=}\DecValTok{1}\OperatorTok{:}\KeywordTok{nrow}\NormalTok{(seal.csv)) }\OperatorTok
\StringTok{   }\KeywordTok{mutate}\NormalTok{(}\DataTypeTok{se_y =} \KeywordTok{sd}\NormalTok{(age.days)}\OperatorTok{/}\StringTok{ }\KeywordTok{sqrt}\NormalTok{(samp_size)) }\OperatorTok
\StringTok{   }\KeywordTok{group_by}\NormalTok{(samp_size) }\OperatorTok
\StringTok{   }\KeywordTok{mutate}\NormalTok{(}\DataTypeTok{z =}\NormalTok{ (age.days}\OperatorTok{-}\NormalTok{s.mean)}\OperatorTok{/}\NormalTok{se_y) }\OperatorTok
\StringTok{   }\KeywordTok{mutate}\NormalTok{(}\DataTypeTok{slope =} \KeywordTok{c}\NormalTok{(length.cm}\OperatorTok{/}\NormalTok{age.days)) }\OperatorTok
\StringTok{   }\KeywordTok{mutate}\NormalTok{(}\DataTypeTok{p =} \DecValTok{2}\OperatorTok{*}\KeywordTok{pnorm}\NormalTok{(}\KeywordTok{abs}\NormalTok{(z), }\DataTypeTok{lower.tail=}\OtherTok{FALSE}\NormalTok{)) }\OperatorTok
\StringTok{   }\KeywordTok{add_residuals}\NormalTok{(seal.lm) }\OperatorTok
\StringTok{   }\KeywordTok{add_predictions}\NormalTok{(seal.lm) }\OperatorTok
\StringTok{   }\KeywordTok{ungroup}\NormalTok{()}

\KeywordTok{max}\NormalTok{(seal.sim}\OperatorTok{$}\NormalTok{slope)}
\end{Highlighting}
\end{Shaded}

\begin{verbatim}
## [1] 0.1200418
\end{verbatim}

\begin{Shaded}
\begin{Highlighting}[]
\KeywordTok{min}\NormalTok{(seal.sim}\OperatorTok{$}\NormalTok{slope)}
\end{Highlighting}
\end{Shaded}

\begin{verbatim}
## [1] 0.01589304
\end{verbatim}

\begin{Shaded}
\begin{Highlighting}[]
\NormalTok{s.slope}
\end{Highlighting}
\end{Shaded}

\begin{verbatim}
## [1] 0.00237
\end{verbatim}

\begin{Shaded}
\begin{Highlighting}[]
\NormalTok{seal.sim.slope<-seal.sim }\OperatorTok
\StringTok{    }\KeywordTok{filter}\NormalTok{(slope }\OperatorTok{<=}\StringTok{ }\NormalTok{s.slope) }\OperatorTok
\StringTok{    }\KeywordTok{ungroup}\NormalTok{()}

\NormalTok{seal.sim.slope<-seal.sim }\OperatorTok
\StringTok{    }\KeywordTok{filter}\NormalTok{(slope }\OperatorTok{>=}\StringTok{ }\FloatTok{0.03}\NormalTok{) }\OperatorTok
\StringTok{    }\KeywordTok{ungroup}\NormalTok{()}

\NormalTok{seal.sim.slope }\OperatorTok
\StringTok{    }\KeywordTok{summarise}\NormalTok{(}\DataTypeTok{power =} \DecValTok{1}\OperatorTok{-}\KeywordTok{sum}\NormalTok{(p}\OperatorTok{>}\FloatTok{0.01}\NormalTok{)}\OperatorTok{/}\KeywordTok{n}\NormalTok{()) }\OperatorTok\StringTok{ }
\StringTok{    }\KeywordTok{ungroup}\NormalTok{()}
\end{Highlighting}
\end{Shaded}

\begin{verbatim}
## # A tibble: 1 x 1
##   power
##   <dbl>
## 1 0.968
\end{verbatim}

\begin{Shaded}
\begin{Highlighting}[]
\NormalTok{seal.sim.slope }\OperatorTok
\StringTok{   }\KeywordTok{summarise}\NormalTok{(}\DataTypeTok{power =} \DecValTok{1}\OperatorTok{-}\KeywordTok{sum}\NormalTok{(p}\OperatorTok{>}\FloatTok{0.05}\NormalTok{)}\OperatorTok{/}\KeywordTok{n}\NormalTok{()) }\OperatorTok\StringTok{ }
\StringTok{   }\KeywordTok{ungroup}\NormalTok{()}
\end{Highlighting}
\end{Shaded}

\begin{verbatim}
## # A tibble: 1 x 1
##   power
##   <dbl>
## 1 0.979
\end{verbatim}

\begin{Shaded}
\begin{Highlighting}[]
\NormalTok{seal.sim.slope }\OperatorTok
\StringTok{   }\KeywordTok{summarise}\NormalTok{(}\DataTypeTok{power =} \DecValTok{1}\OperatorTok{-}\KeywordTok{sum}\NormalTok{(p}\OperatorTok{>}\FloatTok{0.1}\NormalTok{)}\OperatorTok{/}\KeywordTok{n}\NormalTok{()) }\OperatorTok\StringTok{ }
\StringTok{   }\KeywordTok{ungroup}\NormalTok{()}
\end{Highlighting}
\end{Shaded}

\begin{verbatim}
## # A tibble: 1 x 1
##   power
##   <dbl>
## 1 0.985
\end{verbatim}

\begin{Shaded}
\begin{Highlighting}[]
\NormalTok{seal.sim.slope<-seal.sim }\OperatorTok
\StringTok{    }\KeywordTok{filter}\NormalTok{(slope }\OperatorTok{>=}\StringTok{ }\FloatTok{0.05}\NormalTok{) }\OperatorTok
\StringTok{    }\KeywordTok{ungroup}\NormalTok{()}
\NormalTok{seal.sim.slope }\OperatorTok
\StringTok{    }\KeywordTok{summarise}\NormalTok{(}\DataTypeTok{power =} \DecValTok{1}\OperatorTok{-}\KeywordTok{sum}\NormalTok{(p}\OperatorTok{>}\FloatTok{0.01}\NormalTok{)}\OperatorTok{/}\KeywordTok{n}\NormalTok{()) }\OperatorTok\StringTok{ }
\StringTok{    }\KeywordTok{ungroup}\NormalTok{()}
\end{Highlighting}
\end{Shaded}

\begin{verbatim}
## # A tibble: 1 x 1
##   power
##   <dbl>
## 1 0.999
\end{verbatim}

\begin{Shaded}
\begin{Highlighting}[]
\NormalTok{seal.sim.slope }\OperatorTok
\StringTok{   }\KeywordTok{summarise}\NormalTok{(}\DataTypeTok{power =} \DecValTok{1}\OperatorTok{-}\KeywordTok{sum}\NormalTok{(p}\OperatorTok{>}\FloatTok{0.05}\NormalTok{)}\OperatorTok{/}\KeywordTok{n}\NormalTok{()) }\OperatorTok\StringTok{ }
\StringTok{   }\KeywordTok{ungroup}\NormalTok{()}
\end{Highlighting}
\end{Shaded}

\begin{verbatim}
## # A tibble: 1 x 1
##   power
##   <dbl>
## 1 0.999
\end{verbatim}

\begin{Shaded}
\begin{Highlighting}[]
\NormalTok{seal.sim.slope }\OperatorTok
\StringTok{   }\KeywordTok{summarise}\NormalTok{(}\DataTypeTok{power =} \DecValTok{1}\OperatorTok{-}\KeywordTok{sum}\NormalTok{(p}\OperatorTok{>}\FloatTok{0.1}\NormalTok{)}\OperatorTok{/}\KeywordTok{n}\NormalTok{()) }\OperatorTok\StringTok{ }
\StringTok{   }\KeywordTok{ungroup}\NormalTok{()}
\end{Highlighting}
\end{Shaded}

\begin{verbatim}
## # A tibble: 1 x 1
##   power
##   <dbl>
## 1     1
\end{verbatim}

\begin{Shaded}
\begin{Highlighting}[]
\NormalTok{seal.sim.slope<-}\StringTok{ }\NormalTok{seal.sim }\OperatorTok
\StringTok{    }\KeywordTok{filter}\NormalTok{(slope }\OperatorTok{<=}\StringTok{ }\FloatTok{0.075}\NormalTok{) }\OperatorTok
\StringTok{    }\KeywordTok{ungroup}\NormalTok{()}
\NormalTok{seal.sim.slope }\OperatorTok
\StringTok{    }\KeywordTok{summarise}\NormalTok{(}\DataTypeTok{power =} \DecValTok{1}\OperatorTok{-}\KeywordTok{sum}\NormalTok{(p}\OperatorTok{>}\FloatTok{0.01}\NormalTok{)}\OperatorTok{/}\KeywordTok{n}\NormalTok{()) }\OperatorTok\StringTok{ }
\StringTok{    }\KeywordTok{ungroup}\NormalTok{()}
\end{Highlighting}
\end{Shaded}

\begin{verbatim}
## # A tibble: 1 x 1
##   power
##   <dbl>
## 1 0.978
\end{verbatim}

\begin{Shaded}
\begin{Highlighting}[]
\NormalTok{seal.sim.slope }\OperatorTok
\StringTok{   }\KeywordTok{summarise}\NormalTok{(}\DataTypeTok{power =} \DecValTok{1}\OperatorTok{-}\KeywordTok{sum}\NormalTok{(p}\OperatorTok{>}\FloatTok{0.05}\NormalTok{)}\OperatorTok{/}\KeywordTok{n}\NormalTok{()) }\OperatorTok\StringTok{ }
\StringTok{   }\KeywordTok{ungroup}\NormalTok{()}
\end{Highlighting}
\end{Shaded}

\begin{verbatim}
## # A tibble: 1 x 1
##   power
##   <dbl>
## 1 0.986
\end{verbatim}

\begin{Shaded}
\begin{Highlighting}[]
\NormalTok{seal.sim.slope }\OperatorTok
\StringTok{   }\KeywordTok{summarise}\NormalTok{(}\DataTypeTok{power =} \DecValTok{1}\OperatorTok{-}\KeywordTok{sum}\NormalTok{(p}\OperatorTok{>}\FloatTok{0.1}\NormalTok{)}\OperatorTok{/}\KeywordTok{n}\NormalTok{()) }\OperatorTok\StringTok{ }
\StringTok{   }\KeywordTok{ungroup}\NormalTok{()}
\end{Highlighting}
\end{Shaded}

\begin{verbatim}
## # A tibble: 1 x 1
##   power
##   <dbl>
## 1 0.990
\end{verbatim}

\begin{Shaded}
\begin{Highlighting}[]
\NormalTok{seal.sim.slope<-}\StringTok{ }\NormalTok{seal.sim }\OperatorTok
\StringTok{    }\KeywordTok{filter}\NormalTok{(slope }\OperatorTok{<=}\StringTok{ }\FloatTok{0.1}\NormalTok{) }\OperatorTok
\StringTok{    }\KeywordTok{ungroup}\NormalTok{()}
\NormalTok{seal.sim.slope }\OperatorTok
\StringTok{    }\KeywordTok{summarise}\NormalTok{(}\DataTypeTok{power =} \DecValTok{1}\OperatorTok{-}\KeywordTok{sum}\NormalTok{(p}\OperatorTok{>}\FloatTok{0.01}\NormalTok{)}\OperatorTok{/}\KeywordTok{n}\NormalTok{()) }\OperatorTok\StringTok{ }
\StringTok{    }\KeywordTok{ungroup}\NormalTok{()}
\end{Highlighting}
\end{Shaded}

\begin{verbatim}
## # A tibble: 1 x 1
##   power
##   <dbl>
## 1 0.978
\end{verbatim}

\begin{Shaded}
\begin{Highlighting}[]
\NormalTok{seal.sim.slope }\OperatorTok
\StringTok{   }\KeywordTok{summarise}\NormalTok{(}\DataTypeTok{power =} \DecValTok{1}\OperatorTok{-}\KeywordTok{sum}\NormalTok{(p}\OperatorTok{>}\FloatTok{0.05}\NormalTok{)}\OperatorTok{/}\KeywordTok{n}\NormalTok{()) }\OperatorTok\StringTok{ }
\StringTok{   }\KeywordTok{ungroup}\NormalTok{()}
\end{Highlighting}
\end{Shaded}

\begin{verbatim}
## # A tibble: 1 x 1
##   power
##   <dbl>
## 1 0.986
\end{verbatim}

\begin{Shaded}
\begin{Highlighting}[]
\NormalTok{seal.sim.slope }\OperatorTok
\StringTok{   }\KeywordTok{summarise}\NormalTok{(}\DataTypeTok{power =} \DecValTok{1}\OperatorTok{-}\KeywordTok{sum}\NormalTok{(p}\OperatorTok{>}\FloatTok{0.1}\NormalTok{)}\OperatorTok{/}\KeywordTok{n}\NormalTok{()) }\OperatorTok\StringTok{ }
\StringTok{   }\KeywordTok{ungroup}\NormalTok{()}
\end{Highlighting}
\end{Shaded}

\begin{verbatim}
## # A tibble: 1 x 1
##   power
##   <dbl>
## 1 0.990
\end{verbatim}

\begin{quote}
There werent any instantaneous slopes less than the stated value of the
slope. Therefor the effect of slope could only be judged by
incrementally increasing the slope beyond the stated value. Measuring
power of f-test as a function of an increasing slope had a parabolic
relationship. Power increased with increased slope until a maximum was
reached at a slope of 0.05 centimeters/day. As slope increased above
0.05 centimeters/day power began to decrease.
\end{quote}

\section{Extra credit 1 - test whether the distribution of ages alters
power: 3
points}\label{extra-credit-1---test-whether-the-distribution-of-ages-alters-power-3-points}

\subsection{Extra Credit 2 Choose just one of the above elements to
vary. Using likelihood to fit models, repeat your power analysis for a
chi-square likelihood ratio test. You can use glm(), bbmle or some other
means of fitting and obtaining a LRT at your discretion. 5
points.}\label{extra-credit-2-choose-just-one-of-the-above-elements-to-vary.-using-likelihood-to-fit-models-repeat-your-power-analysis-for-a-chi-square-likelihood-ratio-test.-you-can-use-glm-bbmle-or-some-other-means-of-fitting-and-obtaining-a-lrt-at-your-discretion.-5-points.}

\section{4) Bayes Theorem}\label{bayes-theorem}

\begin{quote}
I've referenced the following figure a few times. I'd like you to
demonstrate your understanding of Bayes Theorem by hand showing what the
probability of the sun exploding is given the data. Assume that your
prior probability that the sun explodes is p(Sun Explodes) = 0.0001. The
rest of the information you need is in the cartoon!
\end{quote}

\begin{Shaded}
\begin{Highlighting}[]
\NormalTok{knitr}\OperatorTok{::}\NormalTok{opts_chunk}\OperatorTok{$}\KeywordTok{set}\NormalTok{(}\DataTypeTok{echo =} \OtherTok{TRUE}\NormalTok{)}

\KeywordTok{tibble}\NormalTok{(}\DataTypeTok{pr_not_sixes =} \DecValTok{11}\OperatorTok{/}\DecValTok{12}\NormalTok{,}
       \DataTypeTok{pr_dbl_sixes =} \DecValTok{1}\OperatorTok{/}\DecValTok{12}\NormalTok{,}
       \DataTypeTok{pr_Sun_Explodes =} \FloatTok{.0001}\NormalTok{) }\OperatorTok\StringTok{ }
\StringTok{  }\KeywordTok{mutate}\NormalTok{(}\DataTypeTok{marginal =}\NormalTok{ pr_not_sixes }\OperatorTok{*}\StringTok{ }\NormalTok{pr_Sun_Explodes }\OperatorTok{+}\StringTok{ }\NormalTok{pr_dbl_sixes }\OperatorTok{*}\StringTok{ }\NormalTok{(}\DecValTok{1} \OperatorTok{-}\StringTok{ }\NormalTok{pr_Sun_Explodes)) }\OperatorTok\StringTok{ }
\StringTok{  }\KeywordTok{mutate}\NormalTok{(}\DataTypeTok{pr_sun_boom =}\NormalTok{ pr_not_sixes }\OperatorTok{*}\StringTok{ }\NormalTok{pr_Sun_Explodes }\OperatorTok{/}\StringTok{ }\NormalTok{marginal) }\OperatorTok\StringTok{ }
\StringTok{  }\KeywordTok{mutate}\NormalTok{(}\DataTypeTok{pr_not_boom =}\NormalTok{ pr_dbl_sixes }\OperatorTok{*}\StringTok{ }\NormalTok{(}\DecValTok{1}\OperatorTok{-}\NormalTok{pr_Sun_Explodes) }\OperatorTok{/}\StringTok{ }\NormalTok{marginal) }\OperatorTok\StringTok{ }
\StringTok{  }\KeywordTok{glimpse}\NormalTok{()}
\end{Highlighting}
\end{Shaded}

\begin{verbatim}
## Observations: 1
## Variables: 6
## $ pr_not_sixes    <dbl> 0.9166667
## $ pr_dbl_sixes    <dbl> 0.08333333
## $ pr_Sun_Explodes <dbl> 1e-04
## $ marginal        <dbl> 0.08341667
## $ pr_sun_boom     <dbl> 0.001098901
## $ pr_not_boom     <dbl> 0.9989011
\end{verbatim}

\section{5) Quailing at the Prospect of Linear
Models}\label{quailing-at-the-prospect-of-linear-models}

\begin{quote}
I'd like us to walk through the three different `engines' that we have
learned about to fit linear models. To motivate this, we'll look at
Burness et al.'s 2012 study ``Post-hatch heat warms adult beaks:
irreversible physiological plasticity in Japanese quail
\url{http://rspb.royalsocietypublishing.org/content/280/1767/20131436.short}
the data for which they have made available at Data Dryad at
\url{http://datadryad.org/resource/doi:10.5061/dryad.gs661}. We'll be
looking at the morphology data.
\end{quote}

\section{5.1 Three fits (10 points)}\label{three-fits-10-points}

\begin{quote}
To begin with, I'd like you to fit the relationship that describes how
Tarsus (leg) length predicts upper beak (Culmen) length. Fit this
relationship using least squares, likelihood, and Bayesian techniques.
For each fit, demonstrate that the necessary assumptions have been met.
Note, functions used to fit with likelihood and Bayes may or may not
behave well when fed NAs. So look out for those errors.
\end{quote}

\begin{Shaded}
\begin{Highlighting}[]
\NormalTok{knitr}\OperatorTok{::}\NormalTok{opts_chunk}\OperatorTok{$}\KeywordTok{set}\NormalTok{(}\DataTypeTok{echo =} \OtherTok{TRUE}\NormalTok{)}

\NormalTok{quaildata<-}\KeywordTok{read_csv}\NormalTok{(}\StringTok{"https://datadryad.org/bitstream/handle/10255/dryad.51265/Morphology%20data.csv?sequence=1"}\NormalTok{)}
\end{Highlighting}
\end{Shaded}

\begin{verbatim}
## Parsed with column specification:
## cols(
##   `Bird #` = col_integer(),
##   Sex = col_character(),
##   `Age (days)` = col_integer(),
##   `Exp. Temp. (degree C)` = col_integer(),
##   `Mass (g)` = col_double(),
##   `Tarsus (mm)` = col_double(),
##   `Culmen (mm)` = col_double(),
##   `Depth (mm)` = col_double(),
##   `Width (mm)` = col_double(),
##   NOTES = col_character()
## )
\end{verbatim}

\begin{Shaded}
\begin{Highlighting}[]
\NormalTok{quail.lm <-}\StringTok{ }\KeywordTok{lm}\NormalTok{(quaildata}\OperatorTok{$}\StringTok{`}\DataTypeTok{Culmen (mm)}\StringTok{`} \OperatorTok{~}\StringTok{ }\NormalTok{quaildata}\OperatorTok{$}\StringTok{`}\DataTypeTok{Tarsus (mm)}\StringTok{`}\NormalTok{, }\DataTypeTok{data =}\NormalTok{ quaildata)}
\CommentTok{#assumptions}
\KeywordTok{plot}\NormalTok{(quail.lm, }\DataTypeTok{which=}\DecValTok{1}\NormalTok{)}
\end{Highlighting}
\end{Shaded}

\includegraphics{Midterm_Andrew_Judell-Halfpenny_files/figure-latex/question-5.1-1.pdf}

\begin{Shaded}
\begin{Highlighting}[]
\KeywordTok{plot}\NormalTok{(quail.lm, }\DataTypeTok{which=}\DecValTok{2}\NormalTok{)}
\end{Highlighting}
\end{Shaded}

\includegraphics{Midterm_Andrew_Judell-Halfpenny_files/figure-latex/question-5.1-2.pdf}

\begin{Shaded}
\begin{Highlighting}[]
\NormalTok{quail.mle <-}\StringTok{ }\KeywordTok{glm}\NormalTok{(quaildata}\OperatorTok{$}\StringTok{`}\DataTypeTok{Culmen (mm)}\StringTok{`} \OperatorTok{~}\StringTok{ }\NormalTok{quaildata}\OperatorTok{$}\StringTok{`}\DataTypeTok{Tarsus (mm)}\StringTok{`}\NormalTok{,}
                \DataTypeTok{family =} \KeywordTok{gaussian}\NormalTok{(}\DataTypeTok{link =} \StringTok{"identity"}\NormalTok{),}
                \DataTypeTok{data =}\NormalTok{ quaildata)}

\KeywordTok{plot}\NormalTok{(quail.mle, }\DataTypeTok{which=}\DecValTok{1}\NormalTok{)}
\end{Highlighting}
\end{Shaded}

\includegraphics{Midterm_Andrew_Judell-Halfpenny_files/figure-latex/question-5.1-3.pdf}

\begin{Shaded}
\begin{Highlighting}[]
\KeywordTok{plot}\NormalTok{(quail.mle, }\DataTypeTok{which=}\DecValTok{2}\NormalTok{)}
\end{Highlighting}
\end{Shaded}

\includegraphics{Midterm_Andrew_Judell-Halfpenny_files/figure-latex/question-5.1-4.pdf}

\begin{Shaded}
\begin{Highlighting}[]
\NormalTok{quail.prof <-}\StringTok{ }\KeywordTok{profileModel}\NormalTok{(quail.mle,}
                     \DataTypeTok{objective =} \StringTok{"ordinaryDeviance"}\NormalTok{)}
\end{Highlighting}
\end{Shaded}

\begin{verbatim}
## Profiling for parameter (Intercept) ... Done
## Profiling for parameter quaildata$`Tarsus (mm)` ... Done
\end{verbatim}

\begin{Shaded}
\begin{Highlighting}[]
\KeywordTok{plot}\NormalTok{(quail.prof)}
\end{Highlighting}
\end{Shaded}

\includegraphics{Midterm_Andrew_Judell-Halfpenny_files/figure-latex/question-5.1-5.pdf}

\begin{Shaded}
\begin{Highlighting}[]
\NormalTok{quail.data<-}\KeywordTok{as.data.frame}\NormalTok{(}\KeywordTok{cbind}\NormalTok{(}\DataTypeTok{Culmen=}\NormalTok{quaildata}\OperatorTok{$}\StringTok{`}\DataTypeTok{Culmen (mm)}\StringTok{`}\NormalTok{,}\DataTypeTok{Tarsus=}\NormalTok{quaildata}\OperatorTok{$}\StringTok{`}\DataTypeTok{Tarsus (mm)}\StringTok{`}\NormalTok{))}
\NormalTok{quail.data<-}\KeywordTok{na.omit}\NormalTok{(quail.data)}
\end{Highlighting}
\end{Shaded}

\includegraphics{Midterm_Andrew_Judell-Halfpenny_files/figure-latex/question-5.1.3-1.pdf}
\includegraphics{Midterm_Andrew_Judell-Halfpenny_files/figure-latex/question-5.1.3-2.pdf}

\begin{verbatim}
##                  prior     class   coef group resp dpar nlpar bound
## 1                              b                                   
## 2                              b Tarsus                            
## 3 student_t(3, 12, 10) Intercept                                   
## 4  student_t(3, 0, 10)     sigma
\end{verbatim}

\begin{verbatim}
##   b_Intercept  b_Tarsus    sigma      lp__ chain iter
## 1 -0.16895147 0.3766854 1.227381 -1256.218     1 1001
## 2 -0.20414484 0.3771976 1.276628 -1256.523     1 1002
## 3 -0.09085523 0.3693485 1.207801 -1258.936     1 1003
## 4 -0.15080430 0.3783011 1.228581 -1259.319     1 1004
## 5 -0.08855971 0.3732551 1.221575 -1255.633     1 1005
## 6 -0.13726647 0.3747978 1.221534 -1255.676     1 1006
\end{verbatim}

\includegraphics{Midterm_Andrew_Judell-Halfpenny_files/figure-latex/question-5.1.3-3.pdf}

\section{5.2 Three interpretations (10
points)}\label{three-interpretations-10-points}

\begin{quote}
OK, now that we have fits, take a look! Do the coefficients and their
associated measures of error in their estimation match? How would we
interpret the results from these different analyses differently? Or
would we? Note, confint works on lm objects as well.
\end{quote}

\section{5.3 Everyday I'm Profilin' (10
points)}\label{everyday-im-profilin-10-points}

\begin{quote}
For your likelihood fit, are your profiles well behaved? For just the
slope, use grid sampling to create a profile. You'll need to write
functions for this, and use the results from your glm() fit to provide
the reasonable bounds of what you should be profiling over (3SE should
do). Is it well behaved? Plot the profile and give the 80\% and 95\% CI.
Verify your results with profileModel.
\end{quote}

\begin{Shaded}
\begin{Highlighting}[]
\NormalTok{quallm<-}\KeywordTok{summary}\NormalTok{(quail.lm)}
\NormalTok{quallm}
\end{Highlighting}
\end{Shaded}

\begin{verbatim}
## 
## Call:
## lm(formula = quaildata$`Culmen (mm)` ~ quaildata$`Tarsus (mm)`, 
##     data = quaildata)
## 
## Residuals:
##     Min      1Q  Median      3Q     Max 
## -4.4081 -0.7029 -0.0328  0.7263  3.5970 
## 
## Coefficients:
##                          Estimate Std. Error t value Pr(>|t|)    
## (Intercept)             -0.098707   0.215450  -0.458    0.647    
## quaildata$`Tarsus (mm)`  0.372927   0.006646  56.116   <2e-16 ***
## ---
## Signif. codes:  0 '***' 0.001 '**' 0.01 '*' 0.05 '.' 0.1 ' ' 1
## 
## Residual standard error: 1.238 on 764 degrees of freedom
##   (114 observations deleted due to missingness)
## Multiple R-squared:  0.8048, Adjusted R-squared:  0.8045 
## F-statistic:  3149 on 1 and 764 DF,  p-value: < 2.2e-16
\end{verbatim}

\begin{Shaded}
\begin{Highlighting}[]
\NormalTok{res.sd<-quallm}\OperatorTok{$}\NormalTok{sigma;res.sd}
\end{Highlighting}
\end{Shaded}

\begin{verbatim}
## [1] 1.238383
\end{verbatim}

\begin{Shaded}
\begin{Highlighting}[]
\NormalTok{m.est=quallm}\OperatorTok{$}\NormalTok{coefficients[}\DecValTok{2}\NormalTok{,}\DecValTok{1}\NormalTok{];m.est}
\end{Highlighting}
\end{Shaded}

\begin{verbatim}
## [1] 0.3729268
\end{verbatim}

\begin{Shaded}
\begin{Highlighting}[]
\NormalTok{m.sterr=quallm}\OperatorTok{$}\NormalTok{coefficients[}\DecValTok{2}\NormalTok{,}\DecValTok{2}\NormalTok{];m.sterr}
\end{Highlighting}
\end{Shaded}

\begin{verbatim}
## [1] 0.006645654
\end{verbatim}

\begin{Shaded}
\begin{Highlighting}[]
\NormalTok{b.est=quallm}\OperatorTok{$}\NormalTok{coefficients[}\DecValTok{1}\NormalTok{,}\DecValTok{1}\NormalTok{];b.est}
\end{Highlighting}
\end{Shaded}

\begin{verbatim}
## [1] -0.09870712
\end{verbatim}

\begin{Shaded}
\begin{Highlighting}[]
\NormalTok{b.sterr=quallm}\OperatorTok{$}\NormalTok{coefficients[}\DecValTok{1}\NormalTok{,}\DecValTok{2}\NormalTok{];b.sterr}
\end{Highlighting}
\end{Shaded}

\begin{verbatim}
## [1] 0.2154496
\end{verbatim}

\begin{Shaded}
\begin{Highlighting}[]
\CommentTok{#approx.mod <- glmer(formula = quail.data$Tarsus ~ quail.data$Culmen,}
 \CommentTok{#                   data=quail.data, family=binomial(link="logit"))}


\NormalTok{quail_dist <-}\StringTok{ }\KeywordTok{crossing}\NormalTok{(}\DataTypeTok{b =} \KeywordTok{seq}\NormalTok{(b.est}\DecValTok{-3}\OperatorTok{*}\NormalTok{b.sterr, b.est}\OperatorTok{+}\DecValTok{3}\OperatorTok{*}\NormalTok{b.sterr, b.sterr),}
                      \DataTypeTok{m =} \KeywordTok{seq}\NormalTok{(m.est}\DecValTok{-3}\OperatorTok{*}\NormalTok{m.sterr, m.est}\OperatorTok{+}\DecValTok{3}\OperatorTok{*}\NormalTok{m.sterr, m.sterr),}
                      \DataTypeTok{residual.sd =} \KeywordTok{seq}\NormalTok{(res.sd}\OperatorTok{/}\DecValTok{2}\NormalTok{,res.sd}\OperatorTok{*}\DecValTok{2}\NormalTok{, }\FloatTok{.05}\NormalTok{)) }\OperatorTok
\StringTok{  }\KeywordTok{rowwise}\NormalTok{() }\OperatorTok
\StringTok{  }\KeywordTok{mutate}\NormalTok{(}\DataTypeTok{log_lik =} \KeywordTok{sum}\NormalTok{(}\KeywordTok{dnorm}\NormalTok{(quail.data}\OperatorTok{$}\NormalTok{Tarsus, }\DataTypeTok{mean =}\NormalTok{ m, }\DataTypeTok{sd =}\NormalTok{ residual.sd, }\DataTypeTok{log=}\OtherTok{TRUE}\NormalTok{))) }\OperatorTok
\StringTok{  }\KeywordTok{ungroup}\NormalTok{()}

\NormalTok{quail_dist }\OperatorTok
\StringTok{  }\KeywordTok{filter}\NormalTok{(log_lik }\OperatorTok{==}\StringTok{ }\KeywordTok{max}\NormalTok{(log_lik))}
\end{Highlighting}
\end{Shaded}

\begin{verbatim}
## # A tibble: 7 x 4
##         b     m residual.sd log_lik
##     <dbl> <dbl>       <dbl>   <dbl>
## 1 -0.745  0.393        2.47 -65865.
## 2 -0.530  0.393        2.47 -65865.
## 3 -0.314  0.393        2.47 -65865.
## 4 -0.0987 0.393        2.47 -65865.
## 5  0.117  0.393        2.47 -65865.
## 6  0.332  0.393        2.47 -65865.
## 7  0.548  0.393        2.47 -65865.
\end{verbatim}

\begin{Shaded}
\begin{Highlighting}[]
\NormalTok{quail_sd_profile<-quail_dist }\OperatorTok
\StringTok{  }\KeywordTok{filter}\NormalTok{(log_lik }\OperatorTok{>}\StringTok{ }\KeywordTok{max}\NormalTok{(log_lik) }\OperatorTok{-}\StringTok{ }\FloatTok{1.92}\NormalTok{) }\OperatorTok
\StringTok{    }\KeywordTok{filter}\NormalTok{(}\KeywordTok{row_number}\NormalTok{()}\OperatorTok{==}\DecValTok{1} \OperatorTok{|}\StringTok{ }\KeywordTok{row_number}\NormalTok{()}\OperatorTok{==}\KeywordTok{n}\NormalTok{())}


\NormalTok{fitted <-}\StringTok{ }\KeywordTok{predict}\NormalTok{(quail.mle)}
\NormalTok{res <-}\StringTok{ }\KeywordTok{residuals}\NormalTok{(quail.mle)}
\KeywordTok{qplot}\NormalTok{(fitted, res)}
\end{Highlighting}
\end{Shaded}

\includegraphics{Midterm_Andrew_Judell-Halfpenny_files/figure-latex/question.-5.3-1.pdf}

\begin{Shaded}
\begin{Highlighting}[]
\NormalTok{quail.profile <-}\StringTok{ }\KeywordTok{profileModel}\NormalTok{(quail.mle,}
                     \DataTypeTok{objective =} \StringTok{"ordinaryDeviance"}\NormalTok{,}
                     \DataTypeTok{quantile =} \KeywordTok{qchisq}\NormalTok{(}\FloatTok{0.95}\NormalTok{, }\DecValTok{1}\NormalTok{))}
\end{Highlighting}
\end{Shaded}

\begin{verbatim}
## Preliminary iteration .. Done
## 
## Profiling for parameter (Intercept) ... Done
## Profiling for parameter quaildata$`Tarsus (mm)` ... Done
\end{verbatim}

\begin{Shaded}
\begin{Highlighting}[]
\NormalTok{quail.profile}
\end{Highlighting}
\end{Shaded}

\begin{verbatim}
## 
## Call:  profileModel(fitted = quail.mle, quantile = qchisq(0.95, 1),      objective = "ordinaryDeviance") 
## 
## Profiled parameters:
## [1]  (Intercept)              quaildata$`Tarsus (mm)`
## 
## Asymptotes:
## profileModel has not detected any profiles with asymptotes.
## 
## Quantile was set to: 3.841459 
## Grid size: 20 
## 
## Agreement of the objective with fitting method glm : TRUE 
## Values of the objective less than 1e-08 were considered 0 
## The profile traces are included in the object.
\end{verbatim}

\begin{Shaded}
\begin{Highlighting}[]
\KeywordTok{plot}\NormalTok{(quail.profile)}
\end{Highlighting}
\end{Shaded}

\includegraphics{Midterm_Andrew_Judell-Halfpenny_files/figure-latex/question.-5.3-2.pdf}

\begin{Shaded}
\begin{Highlighting}[]
\NormalTok{likelihood <-}\StringTok{ }\ControlFlowTok{function}\NormalTok{(m, b, residual.sd)\{}
\NormalTok{  fitted <-}\StringTok{ }\NormalTok{b }\OperatorTok{+}\StringTok{ }\NormalTok{m }\OperatorTok{*}\StringTok{ }\NormalTok{quail.data}\OperatorTok{$}\NormalTok{Tarsus}
  \KeywordTok{sum}\NormalTok{(}\KeywordTok{dnorm}\NormalTok{(quail.data}\OperatorTok{$}\NormalTok{Culmen, fitted, residual.sd, }\DataTypeTok{log=}\OtherTok{TRUE}\NormalTok{))\}}

\NormalTok{minimum.likelihood<-}\ControlFlowTok{function}\NormalTok{(m, b, residual.sd) }\DecValTok{-1}\OperatorTok{*}\KeywordTok{likelihood}\NormalTok{(m,b, residual.sd)}

\NormalTok{quail.mle2 <-}\StringTok{ }\KeywordTok{mle2}\NormalTok{(minimum.likelihood, }\DataTypeTok{start =} \KeywordTok{list}\NormalTok{(}\DataTypeTok{m =}\NormalTok{ m.est, }\DataTypeTok{b =}\NormalTok{b.est, }\DataTypeTok{residual.sd =}\NormalTok{res.sd ))}
\end{Highlighting}
\end{Shaded}

\begin{verbatim}
## Warning in dnorm(quail.data$Culmen, fitted, residual.sd, log = TRUE): NaNs
## produced
\end{verbatim}

\begin{Shaded}
\begin{Highlighting}[]
\KeywordTok{summary}\NormalTok{(quail.mle2)}
\end{Highlighting}
\end{Shaded}

\begin{verbatim}
## Maximum likelihood estimation
## 
## Call:
## mle2(minuslogl = minimum.likelihood, start = list(m = m.est, 
##     b = b.est, residual.sd = res.sd))
## 
## Coefficients:
##              Estimate Std. Error z value  Pr(z)    
## m            0.372927   0.006637 56.1891 <2e-16 ***
## b           -0.098707   0.215169 -0.4587 0.6464    
## residual.sd  1.236770   0.031598 39.1406 <2e-16 ***
## ---
## Signif. codes:  0 '***' 0.001 '**' 0.01 '*' 0.05 '.' 0.1 ' ' 1
## 
## -2 log L: 2499.363
\end{verbatim}

\begin{Shaded}
\begin{Highlighting}[]
\KeywordTok{plot}\NormalTok{(}\KeywordTok{profile}\NormalTok{(quail.mle2))}
\end{Highlighting}
\end{Shaded}

\begin{verbatim}
## Warning in dnorm(quail.data$Culmen, fitted, residual.sd, log = TRUE): NaNs
## produced

## Warning in dnorm(quail.data$Culmen, fitted, residual.sd, log = TRUE): NaNs
## produced

## Warning in dnorm(quail.data$Culmen, fitted, residual.sd, log = TRUE): NaNs
## produced

## Warning in dnorm(quail.data$Culmen, fitted, residual.sd, log = TRUE): NaNs
## produced

## Warning in dnorm(quail.data$Culmen, fitted, residual.sd, log = TRUE): NaNs
## produced

## Warning in dnorm(quail.data$Culmen, fitted, residual.sd, log = TRUE): NaNs
## produced

## Warning in dnorm(quail.data$Culmen, fitted, residual.sd, log = TRUE): NaNs
## produced

## Warning in dnorm(quail.data$Culmen, fitted, residual.sd, log = TRUE): NaNs
## produced

## Warning in dnorm(quail.data$Culmen, fitted, residual.sd, log = TRUE): NaNs
## produced

## Warning in dnorm(quail.data$Culmen, fitted, residual.sd, log = TRUE): NaNs
## produced

## Warning in dnorm(quail.data$Culmen, fitted, residual.sd, log = TRUE): NaNs
## produced

## Warning in dnorm(quail.data$Culmen, fitted, residual.sd, log = TRUE): NaNs
## produced

## Warning in dnorm(quail.data$Culmen, fitted, residual.sd, log = TRUE): NaNs
## produced

## Warning in dnorm(quail.data$Culmen, fitted, residual.sd, log = TRUE): NaNs
## produced

## Warning in dnorm(quail.data$Culmen, fitted, residual.sd, log = TRUE): NaNs
## produced

## Warning in dnorm(quail.data$Culmen, fitted, residual.sd, log = TRUE): NaNs
## produced

## Warning in dnorm(quail.data$Culmen, fitted, residual.sd, log = TRUE): NaNs
## produced

## Warning in dnorm(quail.data$Culmen, fitted, residual.sd, log = TRUE): NaNs
## produced

## Warning in dnorm(quail.data$Culmen, fitted, residual.sd, log = TRUE): NaNs
## produced

## Warning in dnorm(quail.data$Culmen, fitted, residual.sd, log = TRUE): NaNs
## produced

## Warning in dnorm(quail.data$Culmen, fitted, residual.sd, log = TRUE): NaNs
## produced

## Warning in dnorm(quail.data$Culmen, fitted, residual.sd, log = TRUE): NaNs
## produced

## Warning in dnorm(quail.data$Culmen, fitted, residual.sd, log = TRUE): NaNs
## produced

## Warning in dnorm(quail.data$Culmen, fitted, residual.sd, log = TRUE): NaNs
## produced

## Warning in dnorm(quail.data$Culmen, fitted, residual.sd, log = TRUE): NaNs
## produced

## Warning in dnorm(quail.data$Culmen, fitted, residual.sd, log = TRUE): NaNs
## produced

## Warning in dnorm(quail.data$Culmen, fitted, residual.sd, log = TRUE): NaNs
## produced

## Warning in dnorm(quail.data$Culmen, fitted, residual.sd, log = TRUE): NaNs
## produced

## Warning in dnorm(quail.data$Culmen, fitted, residual.sd, log = TRUE): NaNs
## produced

## Warning in dnorm(quail.data$Culmen, fitted, residual.sd, log = TRUE): NaNs
## produced

## Warning in dnorm(quail.data$Culmen, fitted, residual.sd, log = TRUE): NaNs
## produced

## Warning in dnorm(quail.data$Culmen, fitted, residual.sd, log = TRUE): NaNs
## produced

## Warning in dnorm(quail.data$Culmen, fitted, residual.sd, log = TRUE): NaNs
## produced

## Warning in dnorm(quail.data$Culmen, fitted, residual.sd, log = TRUE): NaNs
## produced

## Warning in dnorm(quail.data$Culmen, fitted, residual.sd, log = TRUE): NaNs
## produced

## Warning in dnorm(quail.data$Culmen, fitted, residual.sd, log = TRUE): NaNs
## produced

## Warning in dnorm(quail.data$Culmen, fitted, residual.sd, log = TRUE): NaNs
## produced

## Warning in dnorm(quail.data$Culmen, fitted, residual.sd, log = TRUE): NaNs
## produced

## Warning in dnorm(quail.data$Culmen, fitted, residual.sd, log = TRUE): NaNs
## produced

## Warning in dnorm(quail.data$Culmen, fitted, residual.sd, log = TRUE): NaNs
## produced
\end{verbatim}

\includegraphics{Midterm_Andrew_Judell-Halfpenny_files/figure-latex/question.-5.3-3.pdf}

\begin{Shaded}
\begin{Highlighting}[]
\NormalTok{prof95 <-}\StringTok{ }\KeywordTok{profileModel}\NormalTok{(quail.mle,}
                     \DataTypeTok{objective =} \StringTok{"ordinaryDeviance"}\NormalTok{,}
                     \DataTypeTok{quantile =} \KeywordTok{qchisq}\NormalTok{(}\FloatTok{0.95}\NormalTok{, }\DecValTok{1}\NormalTok{))}
\end{Highlighting}
\end{Shaded}

\begin{verbatim}
## Preliminary iteration .. Done
## 
## Profiling for parameter (Intercept) ... Done
## Profiling for parameter quaildata$`Tarsus (mm)` ... Done
\end{verbatim}

\begin{Shaded}
\begin{Highlighting}[]
\KeywordTok{plot}\NormalTok{(prof95)}
\end{Highlighting}
\end{Shaded}

\includegraphics{Midterm_Andrew_Judell-Halfpenny_files/figure-latex/question.-5.3-4.pdf}

\begin{Shaded}
\begin{Highlighting}[]
\NormalTok{prof80 <-}\StringTok{ }\KeywordTok{profileModel}\NormalTok{(quail.mle,}
                     \DataTypeTok{objective =} \StringTok{"ordinaryDeviance"}\NormalTok{,}
                     \DataTypeTok{quantile =} \KeywordTok{qchisq}\NormalTok{(}\FloatTok{0.8}\NormalTok{, }\DecValTok{1}\NormalTok{))}
\end{Highlighting}
\end{Shaded}

\begin{verbatim}
## Preliminary iteration .. Done
## 
## Profiling for parameter (Intercept) ... Done
## Profiling for parameter quaildata$`Tarsus (mm)` ... Done
\end{verbatim}

\begin{Shaded}
\begin{Highlighting}[]
\KeywordTok{plot}\NormalTok{(prof80)}
\end{Highlighting}
\end{Shaded}

\includegraphics{Midterm_Andrew_Judell-Halfpenny_files/figure-latex/question.-5.3-5.pdf}

\section{5.4 The Power of the Prior (10
points)}\label{the-power-of-the-prior-10-points}

\begin{quote}
This data set is pretty big. After excluding NAs in the variables we're
interested in, it's over 766 lines of data! Now, a lot of data can
overhwelm a strong prior. But only to a point. Show first that there is
enough data here that a prior for the slope with an estimate of 0.4 and
a sd of 0.01 is overwhelmed by the data by demonstrating that it
produces similar results to our already fit flat prior. Second, see if a
very small sample size (n = 10) would at least include 0.4 in it's 95\%
Credible Interval. Last, demonstrate at what sample size that 95\% CL
first begins to include 0.4 when we have a strong prior. How much data
do we really need to overcome our prior belief? Note, it takes a long
time to fit these models, so, try a strategy of spacing out the 3-4
sample sizes, and then zoom in on an interesting region.
\end{quote}

\begin{verbatim}
##   b_Intercept  b_Tarsus    sigma      lp__ chain iter
## 1   -11.35940 0.3716594 9.994044 -2963.515     1 1001
## 2   -10.90027 0.3573845 9.997642 -2964.867     1 1002
## 3   -10.49476 0.3440124 9.972296 -2962.699     1 1003
## 4   -10.49252 0.3440027 9.973313 -2962.812     1 1004
## 5   -11.83156 0.3861837 9.954257 -2963.017     1 1005
## 6   -10.96092 0.3583912 9.962484 -2962.708     1 1006
\end{verbatim}

\begin{verbatim}
## Scale for 'colour' is already present. Adding another scale for
## 'colour', which will replace the existing scale.
\end{verbatim}

\includegraphics{Midterm_Andrew_Judell-Halfpenny_files/figure-latex/unnamed-chunk-1-1.pdf}
\includegraphics{Midterm_Andrew_Judell-Halfpenny_files/figure-latex/unnamed-chunk-1-2.pdf}

\section{6. Extra Credit}\label{extra-credit}

\begin{quote}
Make an election forecast as discussed at
\url{https://biol607.github.io/extra.html} - but this isn't just a
winner prediction. 1 point for the correct winner. 5 points for
correctly predicting the popular vote and being within 10\% (3\% just
for trying!). 5 points for predicting the electoral college and geting
no more than 5 states wrong (3 points just for trying). 5 points for
predicting the senate races getting no more than 5 states wrong (3
points just for trying). 1 extra point for each vote percentage within
your 80\% Confidence/Credible Interval. Ditto for the house races.
\end{quote}


\end{document}
