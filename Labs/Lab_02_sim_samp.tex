\documentclass{article}
\usepackage[T1]{fontenc}
\usepackage{Sweave}
\begin{document}
\Sconcordance{concordance:Lab_02_sim_samp.tex:Lab_02_sim_samp.Rnw:%
1 2 1 1 0 11 1 1 2 1 0 1 1 5 0 1 4 1 1 1 6 5 0 1 1 1 4 3 1 4 0 1 3 1 1 %
1 5 14 0 1 11 14 0 1 4 2 0 1 7 5 0 1 2 1 19 18 0 3 1 1 7 27 0 1 3 2 1}

\title{Laboratory 2 - Simulated Sampling}
\author{Andrew Judell-Halfpenny}
\maketitle

%\documentclass[a4paper,titlepage]{tufte-handout}
%\title{ggplot2 Gallery}
%\begin{document}
%\maketitle
%\tableofcontents

\begin{Schunk}
\begin{Sinput}
> library(knitr);library(magrittr); library(markdown); library(dplyr)
> library(ggplot2)
> # cache chunks and do not tidy ggplot2 examples code
> #opts_chunk$set(tidy = FALSE, cache = TRUE)
\end{Sinput}
\end{Schunk}


\begin{Schunk}
\begin{Sinput}
> # r samp
> #There are a number of ways to get random numbers in R from a variety of distributions.
> #For our simulations, let's start with a sample of 40 individuals that's from a population with a mean value of 10, a SD of 3.
> 
> set.seed(323)
> samp <- rnorm(n = 40, mean = 10, sd = 3)
> par(mar = c(4, 4, 1, 1), mgp = c(2, 1, 0), cex = 0.8)
> plot(cars, pch = 20, col = 'darkgray')
> fit <- lm(dist ~ speed, data = cars)
> abline(fit, lwd = 2)
> 
\end{Sinput}
\end{Schunk}


\begin{Schunk}
\begin{Sinput}
> #This is *incredibly* useful in a wide variety of context. For example, let's say we wanted five rows of mtcars.
> 
> # r sample_mtcars
> mtcars[sample(1:nrow(mtcars), 5), ]
\end{Sinput}
\begin{Soutput}
               mpg cyl  disp  hp drat    wt  qsec vs am gear carb
Mazda RX4 Wag 21.0   6 160.0 110 3.90 2.875 17.02  0  1    4    4
Porsche 914-2 26.0   4 120.3  91 4.43 2.140 16.70  0  1    5    2
Toyota Corona 21.5   4 120.1  97 3.70 2.465 20.01  1  0    3    1
Volvo 142E    21.4   4 121.0 109 4.11 2.780 18.60  1  1    4    2
Fiat 128      32.4   4  78.7  66 4.08 2.200 19.47  1  1    4    1
\end{Soutput}
\begin{Sinput}
> # Of course, for that kind of thing, `dplyr` has you covered with
> 
> #r sample_dplyr
> # sample_n(mtcars, 5)
> 
> #  use a porportions - so, here's a random sampling of 10 percent of mtcars
> 
> #r sample_dplyr_frac
> #sample_frac(mtcars, 0.1)
> sample(1:3, 15, replace = TRUE)
\end{Sinput}
\begin{Soutput}
 [1] 3 1 2 2 3 2 1 2 3 3 3 1 3 3 3
\end{Soutput}
\begin{Sinput}
> ### 3.2 Using `group_by()` for simulation
> 
> sampSim <- data.frame(samp_size = rep(3:50, times = 10))
> # Take the data frame
> samp_size <- sampSim %>%
+    rowwise() %>%
+      mutate(samp.mean = mean(rnorm(samp_size,
+                                    mean = 10, sd = 3)),
+      samp_mean_from_sample = mean(sample(samp, size = samp_size, replace=T)))
> plot(samp.mean ~ samp_size, data=samp_size)
> # Also draw a mean from the samp vector with n = samp_size, with replacement
> 
> # For each individual simulation
> 
> 
> #     group_by(sim_number) %>%
> # Use samp_size to get the mean of a random normal population with mean 10 and sd 3
> #  mutate(mean_pop = mean(rnorm(samp_size, mean = 10, sd = 3)),
> 
> # Also draw a mean from the samp vector with n = samp_size, with replacement
> #         mean_sample = mean(sample(samp, size = samp_size, replace=T))) %>%
> # Cleanup
> #    ungroup()
> 
> #### 3.3.1 Faded Examples.
> # Lets try this out, and have you fill in whats missing in these faded examples.
> # r faded_sim, eval=FALSE
> #Some preperatory material
> set.seed(42)
> samp <- rnorm(100, 10, 3)
> sampSim <- data.frame(samp_size = rep(3:50, times = 10))
> sampSim$sim_number = 1:nrow(sampSim)
> #Mean simulations
> sampSim %>%
+   group_by(sim_number) %>%
+   mutate(mean_pop = mean(rnorm(samp_size, mean = 10, sd = 3)),
+          mean_sample = mean(sample(samp, size = samp_size, replace=T))) %>%
+   ungroup()
\end{Sinput}
\begin{Soutput}
# A tibble: 480 x 4
   samp_size sim_number mean_pop mean_sample
       <int>      <int>    <dbl>       <dbl>
 1         3          1    11.2         6.04
 2         4          2    10.6         9.23
 3         5          3    10.2         9.22
 4         6          4     9.58        6.93
 5         7          5     7.83       11.8 
 6         8          6     8.49        9.26
 7         9          7    10.4         9.35
 8        10          8     9.21       11.8 
 9        11          9     9.47        9.37
10        12         10    10.9         9.61
# ... with 470 more rows
\end{Soutput}
\begin{Sinput}
> 
\end{Sinput}
\end{Schunk}


\end{document}
