\documentclass[]{article}
\usepackage{lmodern}
\usepackage{amssymb,amsmath}
\usepackage{ifxetex,ifluatex}
\usepackage{fixltx2e} % provides \textsubscript
\ifnum 0\ifxetex 1\fi\ifluatex 1\fi=0 % if pdftex
  \usepackage[T1]{fontenc}
  \usepackage[utf8]{inputenc}
\else % if luatex or xelatex
  \ifxetex
    \usepackage{mathspec}
  \else
    \usepackage{fontspec}
  \fi
  \defaultfontfeatures{Ligatures=TeX,Scale=MatchLowercase}
\fi
% use upquote if available, for straight quotes in verbatim environments
\IfFileExists{upquote.sty}{\usepackage{upquote}}{}
% use microtype if available
\IfFileExists{microtype.sty}{%
\usepackage{microtype}
\UseMicrotypeSet[protrusion]{basicmath} % disable protrusion for tt fonts
}{}
\usepackage[margin=1in]{geometry}
\usepackage{hyperref}
\hypersetup{unicode=true,
            pdftitle={BIOL-607 Biological Data Analysis Homework 3: Data Visualization},
            pdfauthor={Andrew Judell-Halfpenny},
            pdfborder={0 0 0},
            breaklinks=true}
\urlstyle{same}  % don't use monospace font for urls
\usepackage{color}
\usepackage{fancyvrb}
\newcommand{\VerbBar}{|}
\newcommand{\VERB}{\Verb[commandchars=\\\{\}]}
\DefineVerbatimEnvironment{Highlighting}{Verbatim}{commandchars=\\\{\}}
% Add ',fontsize=\small' for more characters per line
\usepackage{framed}
\definecolor{shadecolor}{RGB}{248,248,248}
\newenvironment{Shaded}{\begin{snugshade}}{\end{snugshade}}
\newcommand{\KeywordTok}[1]{\textcolor[rgb]{0.13,0.29,0.53}{\textbf{#1}}}
\newcommand{\DataTypeTok}[1]{\textcolor[rgb]{0.13,0.29,0.53}{#1}}
\newcommand{\DecValTok}[1]{\textcolor[rgb]{0.00,0.00,0.81}{#1}}
\newcommand{\BaseNTok}[1]{\textcolor[rgb]{0.00,0.00,0.81}{#1}}
\newcommand{\FloatTok}[1]{\textcolor[rgb]{0.00,0.00,0.81}{#1}}
\newcommand{\ConstantTok}[1]{\textcolor[rgb]{0.00,0.00,0.00}{#1}}
\newcommand{\CharTok}[1]{\textcolor[rgb]{0.31,0.60,0.02}{#1}}
\newcommand{\SpecialCharTok}[1]{\textcolor[rgb]{0.00,0.00,0.00}{#1}}
\newcommand{\StringTok}[1]{\textcolor[rgb]{0.31,0.60,0.02}{#1}}
\newcommand{\VerbatimStringTok}[1]{\textcolor[rgb]{0.31,0.60,0.02}{#1}}
\newcommand{\SpecialStringTok}[1]{\textcolor[rgb]{0.31,0.60,0.02}{#1}}
\newcommand{\ImportTok}[1]{#1}
\newcommand{\CommentTok}[1]{\textcolor[rgb]{0.56,0.35,0.01}{\textit{#1}}}
\newcommand{\DocumentationTok}[1]{\textcolor[rgb]{0.56,0.35,0.01}{\textbf{\textit{#1}}}}
\newcommand{\AnnotationTok}[1]{\textcolor[rgb]{0.56,0.35,0.01}{\textbf{\textit{#1}}}}
\newcommand{\CommentVarTok}[1]{\textcolor[rgb]{0.56,0.35,0.01}{\textbf{\textit{#1}}}}
\newcommand{\OtherTok}[1]{\textcolor[rgb]{0.56,0.35,0.01}{#1}}
\newcommand{\FunctionTok}[1]{\textcolor[rgb]{0.00,0.00,0.00}{#1}}
\newcommand{\VariableTok}[1]{\textcolor[rgb]{0.00,0.00,0.00}{#1}}
\newcommand{\ControlFlowTok}[1]{\textcolor[rgb]{0.13,0.29,0.53}{\textbf{#1}}}
\newcommand{\OperatorTok}[1]{\textcolor[rgb]{0.81,0.36,0.00}{\textbf{#1}}}
\newcommand{\BuiltInTok}[1]{#1}
\newcommand{\ExtensionTok}[1]{#1}
\newcommand{\PreprocessorTok}[1]{\textcolor[rgb]{0.56,0.35,0.01}{\textit{#1}}}
\newcommand{\AttributeTok}[1]{\textcolor[rgb]{0.77,0.63,0.00}{#1}}
\newcommand{\RegionMarkerTok}[1]{#1}
\newcommand{\InformationTok}[1]{\textcolor[rgb]{0.56,0.35,0.01}{\textbf{\textit{#1}}}}
\newcommand{\WarningTok}[1]{\textcolor[rgb]{0.56,0.35,0.01}{\textbf{\textit{#1}}}}
\newcommand{\AlertTok}[1]{\textcolor[rgb]{0.94,0.16,0.16}{#1}}
\newcommand{\ErrorTok}[1]{\textcolor[rgb]{0.64,0.00,0.00}{\textbf{#1}}}
\newcommand{\NormalTok}[1]{#1}
\usepackage{graphicx,grffile}
\makeatletter
\def\maxwidth{\ifdim\Gin@nat@width>\linewidth\linewidth\else\Gin@nat@width\fi}
\def\maxheight{\ifdim\Gin@nat@height>\textheight\textheight\else\Gin@nat@height\fi}
\makeatother
% Scale images if necessary, so that they will not overflow the page
% margins by default, and it is still possible to overwrite the defaults
% using explicit options in \includegraphics[width, height, ...]{}
\setkeys{Gin}{width=\maxwidth,height=\maxheight,keepaspectratio}
\IfFileExists{parskip.sty}{%
\usepackage{parskip}
}{% else
\setlength{\parindent}{0pt}
\setlength{\parskip}{6pt plus 2pt minus 1pt}
}
\setlength{\emergencystretch}{3em}  % prevent overfull lines
\providecommand{\tightlist}{%
  \setlength{\itemsep}{0pt}\setlength{\parskip}{0pt}}
\setcounter{secnumdepth}{0}
% Redefines (sub)paragraphs to behave more like sections
\ifx\paragraph\undefined\else
\let\oldparagraph\paragraph
\renewcommand{\paragraph}[1]{\oldparagraph{#1}\mbox{}}
\fi
\ifx\subparagraph\undefined\else
\let\oldsubparagraph\subparagraph
\renewcommand{\subparagraph}[1]{\oldsubparagraph{#1}\mbox{}}
\fi

%%% Use protect on footnotes to avoid problems with footnotes in titles
\let\rmarkdownfootnote\footnote%
\def\footnote{\protect\rmarkdownfootnote}

%%% Change title format to be more compact
\usepackage{titling}

% Create subtitle command for use in maketitle
\newcommand{\subtitle}[1]{
  \posttitle{
    \begin{center}\large#1\end{center}
    }
}

\setlength{\droptitle}{-2em}

  \title{BIOL-607 Biological Data Analysis Homework 3: Data Visualization}
    \pretitle{\vspace{\droptitle}\centering\huge}
  \posttitle{\par}
    \author{Andrew Judell-Halfpenny}
    \preauthor{\centering\large\emph}
  \postauthor{\par}
      \predate{\centering\large\emph}
  \postdate{\par}
    \date{September 27, 2018}


\begin{document}
\maketitle

\begin{center}\rule{0.5\linewidth}{\linethickness}\end{center}

\subsection{Libraries Used}\label{libraries-used}

\begin{Shaded}
\begin{Highlighting}[]
\CommentTok{# load the libraries and the abd dat}
\KeywordTok{library}\NormalTok{(devtools)}
\CommentTok{#devtools::install_github(gganimate)}
\CommentTok{#devtools::install_github(phangorn)}
\CommentTok{#devtools::install_github(treeio)}
\CommentTok{#library(ggvis);library(phangorn)}
\CommentTok{#library(ggtree);library(treeio);}

\CommentTok{#1. load libraries}
\KeywordTok{library}\NormalTok{(tidyr);}\KeywordTok{library}\NormalTok{(tidygraph);}\KeywordTok{library}\NormalTok{(lubridate)}
\end{Highlighting}
\end{Shaded}

\begin{verbatim}
## 
## Attaching package: 'tidygraph'
\end{verbatim}

\begin{verbatim}
## The following object is masked from 'package:stats':
## 
##     filter
\end{verbatim}

\begin{verbatim}
## 
## Attaching package: 'lubridate'
\end{verbatim}

\begin{verbatim}
## The following object is masked from 'package:base':
## 
##     date
\end{verbatim}

\begin{Shaded}
\begin{Highlighting}[]
\KeywordTok{library}\NormalTok{(ggplot2);}\KeywordTok{library}\NormalTok{(animation);}\KeywordTok{library}\NormalTok{(readr)}
\KeywordTok{library}\NormalTok{(abd);}\KeywordTok{library}\NormalTok{(dplyr);}\KeywordTok{library}\NormalTok{(magrittr)}
\end{Highlighting}
\end{Shaded}

\begin{verbatim}
## Loading required package: nlme
\end{verbatim}

\begin{verbatim}
## Loading required package: lattice
\end{verbatim}

\begin{verbatim}
## Loading required package: grid
\end{verbatim}

\begin{verbatim}
## Loading required package: mosaic
\end{verbatim}

\begin{verbatim}
## Loading required package: dplyr
\end{verbatim}

\begin{verbatim}
## 
## Attaching package: 'dplyr'
\end{verbatim}

\begin{verbatim}
## The following object is masked from 'package:nlme':
## 
##     collapse
\end{verbatim}

\begin{verbatim}
## The following objects are masked from 'package:lubridate':
## 
##     intersect, setdiff, union
\end{verbatim}

\begin{verbatim}
## The following objects are masked from 'package:stats':
## 
##     filter, lag
\end{verbatim}

\begin{verbatim}
## The following objects are masked from 'package:base':
## 
##     intersect, setdiff, setequal, union
\end{verbatim}

\begin{verbatim}
## Loading required package: ggformula
\end{verbatim}

\begin{verbatim}
## Loading required package: ggstance
\end{verbatim}

\begin{verbatim}
## 
## Attaching package: 'ggstance'
\end{verbatim}

\begin{verbatim}
## The following objects are masked from 'package:ggplot2':
## 
##     geom_errorbarh, GeomErrorbarh
\end{verbatim}

\begin{verbatim}
## 
## New to ggformula?  Try the tutorials: 
##  learnr::run_tutorial("introduction", package = "ggformula")
##  learnr::run_tutorial("refining", package = "ggformula")
\end{verbatim}

\begin{verbatim}
## Loading required package: mosaicData
\end{verbatim}

\begin{verbatim}
## Loading required package: Matrix
\end{verbatim}

\begin{verbatim}
## 
## Attaching package: 'Matrix'
\end{verbatim}

\begin{verbatim}
## The following object is masked from 'package:tidyr':
## 
##     expand
\end{verbatim}

\begin{verbatim}
## 
## The 'mosaic' package masks several functions from core packages in order to add 
## additional features.  The original behavior of these functions should not be affected by this.
## 
## Note: If you use the Matrix package, be sure to load it BEFORE loading mosaic.
\end{verbatim}

\begin{verbatim}
## 
## Attaching package: 'mosaic'
\end{verbatim}

\begin{verbatim}
## The following object is masked from 'package:Matrix':
## 
##     mean
\end{verbatim}

\begin{verbatim}
## The following objects are masked from 'package:dplyr':
## 
##     count, do, tally
\end{verbatim}

\begin{verbatim}
## The following object is masked from 'package:ggplot2':
## 
##     stat
\end{verbatim}

\begin{verbatim}
## The following objects are masked from 'package:stats':
## 
##     binom.test, cor, cor.test, cov, fivenum, IQR, median,
##     prop.test, quantile, sd, t.test, var
\end{verbatim}

\begin{verbatim}
## The following objects are masked from 'package:base':
## 
##     max, mean, min, prod, range, sample, sum
\end{verbatim}

\begin{verbatim}
## 
## Attaching package: 'magrittr'
\end{verbatim}

\begin{verbatim}
## The following object is masked from 'package:tidyr':
## 
##     extract
\end{verbatim}

\begin{Shaded}
\begin{Highlighting}[]
\KeywordTok{library}\NormalTok{(forcats)}
\end{Highlighting}
\end{Shaded}

\subsection{For the first three problems, consider the data from problem
9 on page
109}\label{for-the-first-three-problems-consider-the-data-from-problem-9-on-page-109}

\subsubsection{1) Complete problems 10 and 17-18 on pg. 109-111. Use R
where possible. Using an approximate method, provide a rough 95\%
confidence interval for the population
mean.}\label{complete-problems-10-and-17-18-on-pg.-109-111.-use-r-where-possible.-using-an-approximate-method-provide-a-rough-95-confidence-interval-for-the-population-mean.}

\begin{Shaded}
\begin{Highlighting}[]
\NormalTok{abd.genes<-}\KeywordTok{read.csv}\NormalTok{(}\KeywordTok{url}\NormalTok{(}\StringTok{"http://whitlockschluter.zoology.ubc.ca/wp-content/data/chapter04/chap04e1HumanGeneLengths.csv"}\NormalTok{))}
\NormalTok{abd.genes<-}\KeywordTok{as.matrix}\NormalTok{(abd.genes)}

\CommentTok{# standard error function}
\NormalTok{std.er <-}\StringTok{ }\ControlFlowTok{function}\NormalTok{(x)\{}
\NormalTok{  output <-}\KeywordTok{sd}\NormalTok{(x)}\OperatorTok{/}\KeywordTok{sqrt}\NormalTok{(}\KeywordTok{length}\NormalTok{(x))}
  \KeywordTok{return}\NormalTok{(output)\}}
\NormalTok{abd.std.er<-}\KeywordTok{std.er}\NormalTok{(abd.genes)}
\NormalTok{abd.mean<-}\KeywordTok{mean}\NormalTok{(abd.genes)}

\CommentTok{# This is the standard eror variable}
\NormalTok{abd.std.er}
\end{Highlighting}
\end{Shaded}

\begin{verbatim}
## [1] 14.30023
\end{verbatim}

\begin{Shaded}
\begin{Highlighting}[]
\CommentTok{# This is the sample mean}
\NormalTok{abd.mean}
\end{Highlighting}
\end{Shaded}

\begin{verbatim}
## [1] 2622.027
\end{verbatim}

\begin{Shaded}
\begin{Highlighting}[]
\NormalTok{l.abs.genes=abd.mean }\OperatorTok{-}\StringTok{ }\DecValTok{2} \OperatorTok{*}\StringTok{ }\NormalTok{abd.std.er}
\NormalTok{h.abs.genes=abd.mean }\OperatorTok{+}\StringTok{ }\DecValTok{2} \OperatorTok{*}\StringTok{ }\NormalTok{abd.std.er}

\CommentTok{# This variable serves as the lower bound of the 95% confidence interval}
\NormalTok{l.abs.genes}
\end{Highlighting}
\end{Shaded}

\begin{verbatim}
## [1] 2593.427
\end{verbatim}

\begin{Shaded}
\begin{Highlighting}[]
\CommentTok{# This variable serves as the upper bound of the 95% confidence interval}
\NormalTok{h.abs.genes}
\end{Highlighting}
\end{Shaded}

\begin{verbatim}
## [1] 2650.628
\end{verbatim}

\subsubsection{1 b) Provide an interpretation of the interval you
calculated.}\label{b-provide-an-interpretation-of-the-interval-you-calculated.}

\subsection{Answer:}\label{answer}

\subsubsection{The confidence interval that was just calculated is the
range of range of values that we can be 95\% certain holds the value of
the true mean of the sample
populatio.}\label{the-confidence-interval-that-was-just-calculated-is-the-range-of-range-of-values-that-we-can-be-95-certain-holds-the-value-of-the-true-mean-of-the-sample-populatio.}

\subsubsection{The following figure is from the website of a U.S.
national environmental laboratory.7 It displays sample mean
concentrations, with 95\% confidence intervals, of three
radioactive}\label{the-following-figure-is-from-the-website-of-a-u.s.-national-environmental-laboratory.7-it-displays-sample-mean-concentrations-with-95-confidence-intervals-of-three-radioactive}

\subsubsection{\texorpdfstring{substances. The text accompanying the
figure explained that ``the first plotted mean is 2.0 ± 1.1, so there is
a 95\% chance that the actual result is between 0.9 and 3.1, a 2.5\%
chance \# it is less than 0.9, and a 2.5\% chance it is greater than
3.1.'' Is this a correct interpretation of a confidence interval?
Explain.}{substances. The text accompanying the figure explained that the first plotted mean is 2.0 ± 1.1, so there is a 95\% chance that the actual result is between 0.9 and 3.1, a 2.5\% chance \# it is less than 0.9, and a 2.5\% chance it is greater than 3.1. Is this a correct interpretation of a confidence interval? Explain.}}\label{substances.-the-text-accompanying-the-figure-explained-that-the-first-plotted-mean-is-2.0-1.1-so-there-is-a-95-chance-that-the-actual-result-is-between-0.9-and-3.1-a-2.5-chance-it-is-less-than-0.9-and-a-2.5-chance-it-is-greater-than-3.1.-is-this-a-correct-interpretation-of-a-confidence-interval-explain.}

\subsection{Answer:}\label{answer-1}

\subsubsection{No this is statement belies a misunderstanding of the
nature of a confidence interval. The 95\% confidence interval is not
integrated over the area of a probability densityy function therefor it
is not possible to calculate the probability that the true population
parameter mean for a subset or slice of the original
field.}\label{no-this-is-statement-belies-a-misunderstanding-of-the-nature-of-a-confidence-interval.-the-95-confidence-interval-is-not-integrated-over-the-area-of-a-probability-densityy-function-therefor-it-is-not-possible-to-calculate-the-probability-that-the-true-population-parameter-mean-for-a-subset-or-slice-of-the-original-field.}

\subsubsection{\texorpdfstring{18 Amorphophallus johnsonii is a plant
growing in West Africa, and it is better known as a ``corpse-flower.''
Its common name comes from the fact that when it flowers, it gives off a
``powerful aroma of rotting fish and faeces'' (Beath 1996). The flowers
smell this way because their principal pollinators are carrion beetles,
who are attracted to such a smell. Beath (1996) observed the number of
carrion beetles (Phaeochrous amplus) that arrive per night to flowers of
this species. The data are as
follows:}{18 Amorphophallus johnsonii is a plant growing in West Africa, and it is better known as a corpse-flower. Its common name comes from the fact that when it flowers, it gives off a powerful aroma of rotting fish and faeces (Beath 1996). The flowers smell this way because their principal pollinators are carrion beetles, who are attracted to such a smell. Beath (1996) observed the number of carrion beetles (Phaeochrous amplus) that arrive per night to flowers of this species. The data are as follows:}}\label{amorphophallus-johnsonii-is-a-plant-growing-in-west-africa-and-it-is-better-known-as-a-corpse-flower.-its-common-name-comes-from-the-fact-that-when-it-flowers-it-gives-off-a-powerful-aroma-of-rotting-fish-and-faeces-beath-1996.-the-flowers-smell-this-way-because-their-principal-pollinators-are-carrion-beetles-who-are-attracted-to-such-a-smell.-beath-1996-observed-the-number-of-carrion-beetles-phaeochrous-amplus-that-arrive-per-night-to-flowers-of-this-species.-the-data-are-as-follows}

\begin{Shaded}
\begin{Highlighting}[]
\NormalTok{dung_visits<-}\KeywordTok{c}\NormalTok{(}\DecValTok{51}\NormalTok{, }\DecValTok{45}\NormalTok{, }\DecValTok{61}\NormalTok{, }\DecValTok{76}\NormalTok{, }\DecValTok{11}\NormalTok{, }\DecValTok{117}\NormalTok{, }\DecValTok{7}\NormalTok{, }\DecValTok{132}\NormalTok{, }\DecValTok{52}\NormalTok{, }\DecValTok{149}\NormalTok{)}
\NormalTok{dung_visits<-}\KeywordTok{as.matrix}\NormalTok{(dung_visits)}
\end{Highlighting}
\end{Shaded}

\subsubsection{18 a) What is the mean and standard deviation of beetles
per
flower?}\label{a-what-is-the-mean-and-standard-deviation-of-beetles-per-flower}

\begin{Shaded}
\begin{Highlighting}[]
\NormalTok{mean_dung_vists<-}\KeywordTok{mean}\NormalTok{(dung_visits)}
\NormalTok{mean_dung_vists}
\end{Highlighting}
\end{Shaded}

\begin{verbatim}
## [1] 70.1
\end{verbatim}

\subsubsection{18 b) What is the standard error of this estimate of the
mean?}\label{b-what-is-the-standard-error-of-this-estimate-of-the-mean}

\begin{Shaded}
\begin{Highlighting}[]
\KeywordTok{sd}\NormalTok{(dung_visits)}
\end{Highlighting}
\end{Shaded}

\begin{verbatim}
## [1] 48.50074
\end{verbatim}

\begin{Shaded}
\begin{Highlighting}[]
\KeywordTok{str}\NormalTok{(dung_visits)}
\end{Highlighting}
\end{Shaded}

\begin{verbatim}
##  num [1:10, 1] 51 45 61 76 11 117 7 132 52 149
\end{verbatim}

\begin{Shaded}
\begin{Highlighting}[]
\NormalTok{visit.st.er<-}\KeywordTok{std.er}\NormalTok{(dung_visits)}
\end{Highlighting}
\end{Shaded}

\subsubsection{18 c) Give an approximate 95\% confidence interval of the
mean. Provide lower and upper
limits.}\label{c-give-an-approximate-95-confidence-interval-of-the-mean.-provide-lower-and-upper-limits.}

\begin{Shaded}
\begin{Highlighting}[]
\NormalTok{mean_dung_vists }\OperatorTok{+}\StringTok{ }\DecValTok{2} \OperatorTok{*}\StringTok{ }\NormalTok{visit.st.er}
\end{Highlighting}
\end{Shaded}

\begin{verbatim}
## [1] 100.7746
\end{verbatim}

\begin{Shaded}
\begin{Highlighting}[]
\NormalTok{mean_dung_vists }\OperatorTok{-}\StringTok{ }\DecValTok{2} \OperatorTok{*}\StringTok{ }\NormalTok{visit.st.er}
\end{Highlighting}
\end{Shaded}

\begin{verbatim}
## [1] 39.42544
\end{verbatim}

\subsubsection{18 d) If you had been given 25 data points instead of 10,
would you expect the mean to be greater, less than, or about the same as
the mean of this
sample?}\label{d-if-you-had-been-given-25-data-points-instead-of-10-would-you-expect-the-mean-to-be-greater-less-than-or-about-the-same-as-the-mean-of-this-sample}

\subsection{Answer: I expect that the walue be identified as the mean
could potentially change not necessarilly in a particular direction. I
assume with greater sample size the calculation of the mean would be ed
mean approaches the vb but the measurement closer to the paramter that
is the mean
of}\label{answer-i-expect-that-the-walue-be-identified-as-the-mean-could-potentially-change-not-necessarilly-in-a-particular-direction.-i-assume-with-greater-sample-size-the-calculation-of-the-mean-would-be-ed-mean-approaches-the-vb-but-the-measurement-closer-to-the-paramter-that-is-the-mean-of}

\subsubsection{18 e) If you had been given 25 data points instead of 10,
would you have expected the standard deviation to be greater, less than,
or about the same as this
sample?}\label{e-if-you-had-been-given-25-data-points-instead-of-10-would-you-have-expected-the-standard-deviation-to-be-greater-less-than-or-about-the-same-as-this-sample}

\subsection{Answer: I expect that with aa greater sample size the
estimate ofthe standard deviation wouuld be slightly different due to
stochasticirty of the valculation but booth the mean and stanard
deviation}\label{answer-i-expect-that-with-aa-greater-sample-size-the-estimate-ofthe-standard-deviation-wouuld-be-slightly-different-due-to-stochasticirty-of-the-valculation-but-booth-the-mean-and-stanard-deviation}

\subsubsection{area independent of the size of a
sample.}\label{area-independent-of-the-size-of-a-sample.}

\subsubsection{18 f) If you had been given 25 data points instead of 10,
would you have expected the standard error to be greater, less than, or
about the same as this
sample?}\label{f-if-you-had-been-given-25-data-points-instead-of-10-would-you-have-expected-the-standard-error-to-be-greater-less-than-or-about-the-same-as-this-sample}

\subsection{Answer: The standard error can be expected to decrease after
resampling with a grater population. This is because with a larger
sample the calculation of the true mean is more precice. The true meaan
and standard
deviation.}\label{answer-the-standard-error-can-be-expected-to-decrease-after-resampling-with-a-grater-population.-this-is-because-with-a-larger-sample-the-calculation-of-the-true-mean-is-more-precice.-the-true-meaan-and-standard-deviation.}

\subsubsection{2.1) Load the data using readr and make the Month\_Names
column into a factor whose levels are in order of month using
forcats::fct\_inorder. Use levels() - a function that takes a factor
vector and returns the unique levels - on the column. Are they in the
right
order?}\label{load-the-data-using-readr-and-make-the-month_names-column-into-a-factor-whose-levels-are-in-order-of-month-using-forcatsfct_inorder.-use-levels---a-function-that-takes-a-factor-vector-and-returns-the-unique-levels---on-the-column.-are-they-in-the-right-order}

\begin{Shaded}
\begin{Highlighting}[]
\NormalTok{NHice<-readr}\OperatorTok{::}\KeywordTok{read_csv}\NormalTok{(}\StringTok{"/home/drew/Downloads/BIOL607-Biological-Data-Analysis/week-03/NH_seaice_extent_monthly_1978_2016.csv"}\NormalTok{)}
\end{Highlighting}
\end{Shaded}

\begin{verbatim}
## Parsed with column specification:
## cols(
##   Year = col_integer(),
##   Month = col_integer(),
##   Day = col_integer(),
##   Extent = col_double(),
##   Missing = col_integer(),
##   Month_Name = col_character()
## )
\end{verbatim}

\begin{Shaded}
\begin{Highlighting}[]
\NormalTok{NHice}\OperatorTok{$}\NormalTok{monthfactor<-}\KeywordTok{factor}\NormalTok{(NHice}\OperatorTok{$}\NormalTok{Month_Name,}\DataTypeTok{levels=}\NormalTok{month.abb, }\DataTypeTok{ordered=}\OtherTok{TRUE}\NormalTok{)}
\KeywordTok{levels}\NormalTok{(NHice}\OperatorTok{$}\NormalTok{monthfactor)}
\end{Highlighting}
\end{Shaded}

\begin{verbatim}
##  [1] "Jan" "Feb" "Mar" "Apr" "May" "Jun" "Jul" "Aug" "Sep" "Oct" "Nov"
## [12] "Dec"
\end{verbatim}

\begin{Shaded}
\begin{Highlighting}[]
\NormalTok{NHice}\OperatorTok{$}\NormalTok{inorder<-}\KeywordTok{fct_inorder}\NormalTok{(NHice}\OperatorTok{$}\NormalTok{monthfactor, }\DataTypeTok{ordered=}\OtherTok{TRUE}\NormalTok{)}
\KeywordTok{levels}\NormalTok{(NHice}\OperatorTok{$}\NormalTok{inorder)}
\end{Highlighting}
\end{Shaded}

\begin{verbatim}
##  [1] "Nov" "Dec" "Feb" "Mar" "Jun" "Jul" "Sep" "Oct" "Jan" "Apr" "May"
## [12] "Aug"
\end{verbatim}

\subsubsection{2.2) Try fct\_rev() on ice\$Month\_Name (I called my data
frame ice when I made this). What is the order of factor levels that
results? Try out fct\_relevel(), and last, fct\_recode() as well. Look
at the help files to learn more, and in particular try out the examples.
Use these to guide how you try each functino out. After trying each of
these, mutate month name to get the months in the right order, from
January to December. Show that it worked with
levels()}\label{try-fct_rev-on-icemonth_name-i-called-my-data-frame-ice-when-i-made-this.-what-is-the-order-of-factor-levels-that-results-try-out-fct_relevel-and-last-fct_recode-as-well.-look-at-the-help-files-to-learn-more-and-in-particular-try-out-the-examples.-use-these-to-guide-how-you-try-each-functino-out.-after-trying-each-of-these-mutate-month-name-to-get-the-months-in-the-right-order-from-january-to-december.-show-that-it-worked-with-levels}

\begin{Shaded}
\begin{Highlighting}[]
\KeywordTok{levels}\NormalTok{(}\KeywordTok{fct_rev}\NormalTok{(NHice}\OperatorTok{$}\NormalTok{Month_Name))}
\end{Highlighting}
\end{Shaded}

\begin{verbatim}
##  [1] "Sep" "Oct" "Nov" "May" "Mar" "Jun" "Jul" "Jan" "Feb" "Dec" "Aug"
## [12] "Apr"
\end{verbatim}

\begin{Shaded}
\begin{Highlighting}[]
\KeywordTok{levels}\NormalTok{(}\KeywordTok{fct_relevel}\NormalTok{(NHice}\OperatorTok{$}\NormalTok{Month_Name))}
\end{Highlighting}
\end{Shaded}

\begin{verbatim}
##  [1] "Apr" "Aug" "Dec" "Feb" "Jan" "Jul" "Jun" "Mar" "May" "Nov" "Oct"
## [12] "Sep"
\end{verbatim}

\begin{Shaded}
\begin{Highlighting}[]
\KeywordTok{levels}\NormalTok{(}\KeywordTok{fct_recode}\NormalTok{(NHice}\OperatorTok{$}\NormalTok{Month_Name))}
\end{Highlighting}
\end{Shaded}

\begin{verbatim}
##  [1] "Apr" "Aug" "Dec" "Feb" "Jan" "Jul" "Jun" "Mar" "May" "Nov" "Oct"
## [12] "Sep"
\end{verbatim}

\begin{Shaded}
\begin{Highlighting}[]
\NormalTok{NHice}\OperatorTok{$}\NormalTok{monthfactor<-}\KeywordTok{factor}\NormalTok{(NHice}\OperatorTok{$}\NormalTok{Month_Name,}\DataTypeTok{levels=}\NormalTok{month.abb, }\DataTypeTok{ordered=}\OtherTok{TRUE}\NormalTok{)}
\KeywordTok{levels}\NormalTok{(NHice}\OperatorTok{$}\NormalTok{monthfactor)}
\end{Highlighting}
\end{Shaded}

\begin{verbatim}
##  [1] "Jan" "Feb" "Mar" "Apr" "May" "Jun" "Jul" "Aug" "Sep" "Oct" "Nov"
## [12] "Dec"
\end{verbatim}

\begin{Shaded}
\begin{Highlighting}[]
\KeywordTok{head}\NormalTok{(NHice)}
\end{Highlighting}
\end{Shaded}

\begin{verbatim}
## # A tibble: 6 x 8
##    Year Month   Day Extent Missing Month_Name monthfactor inorder
##   <int> <int> <int>  <dbl>   <int> <chr>      <ord>       <ord>  
## 1  1978    11     1   10.7       0 Nov        Nov         Nov    
## 2  1978    12     1   12.7       0 Dec        Dec         Dec    
## 3  1979     2     1   15.9       0 Feb        Feb         Feb    
## 4  1979     3     1   16.6       0 Mar        Mar         Mar    
## 5  1979     6     1   13.1       0 Jun        Jun         Jun    
## 6  1979     7     1   11.6       0 Jul        Jul         Jul
\end{verbatim}

\subsubsection{2.3) Now, using what you have just learned about forcats,
make a column called Season that is a copy of Month\_Name. Use the
function fct\_recode to turn it into a factor vector with the levels
Winter, Spring, Summer, Fall in that order. Use levels() on ice\$Season
to show that it
worked.}\label{now-using-what-you-have-just-learned-about-forcats-make-a-column-called-season-that-is-a-copy-of-month_name.-use-the-function-fct_recode-to-turn-it-into-a-factor-vector-with-the-levels-winter-spring-summer-fall-in-that-order.-use-levels-on-iceseason-to-show-that-it-worked.}

\begin{Shaded}
\begin{Highlighting}[]
\NormalTok{NHice}\OperatorTok{$}\NormalTok{season<-}\KeywordTok{fct_collapse}\NormalTok{(NHice}\OperatorTok{$}\NormalTok{monthfactor,}\DataTypeTok{Winter=}\KeywordTok{c}\NormalTok{(}\StringTok{"Dec"}\NormalTok{,}\StringTok{"Jan"}\NormalTok{, }\StringTok{"Feb"}\NormalTok{),}\DataTypeTok{Spring=}\KeywordTok{c}\NormalTok{(}\StringTok{"Mar"}\NormalTok{,}\StringTok{"Apr"}\NormalTok{, }\StringTok{"May"}\NormalTok{), }\DataTypeTok{Summer=}\KeywordTok{c}\NormalTok{(}\StringTok{"Jun"}\NormalTok{, }\StringTok{"Jul"}\NormalTok{, }\StringTok{"Aug"}\NormalTok{), }\DataTypeTok{Fall=}\KeywordTok{c}\NormalTok{(}\StringTok{"Sep"}\NormalTok{, }\StringTok{"Oct"}\NormalTok{, }\StringTok{"Nov"}\NormalTok{))}
\KeywordTok{levels}\NormalTok{(NHice}\OperatorTok{$}\NormalTok{season)}
\end{Highlighting}
\end{Shaded}

\begin{verbatim}
## [1] "Winter" "Spring" "Summer" "Fall"
\end{verbatim}

\subsubsection{2.4) Make a boxplot showing the variability in sea ice
extent every
month.}\label{make-a-boxplot-showing-the-variability-in-sea-ice-extent-every-month.}

\begin{Shaded}
\begin{Highlighting}[]
\KeywordTok{boxplot}\NormalTok{(Extent }\OperatorTok{~}\StringTok{ }\NormalTok{Month_Name, }\DataTypeTok{data =}\NormalTok{ NHice, }\DataTypeTok{col =}\NormalTok{ NHice}\OperatorTok{$}\NormalTok{Month,}\DataTypeTok{main=}\StringTok{"Variability of Sea Ice Extent by Month"}\NormalTok{)}
\end{Highlighting}
\end{Shaded}

\includegraphics{HW_03_Judell_Halfpenny_Andrew_2018_files/figure-latex/unnamed-chunk-10-1.pdf}

\subsubsection{2.5) With the original data, plot sea ice by year, with
different lines for different months. Then, use facet\_wrap and
cut\_interval(Month, n=4) to split the plot into
seasons.}\label{with-the-original-data-plot-sea-ice-by-year-with-different-lines-for-different-months.-then-use-facet_wrap-and-cut_intervalmonth-n4-to-split-the-plot-into-seasons.}

\begin{Shaded}
\begin{Highlighting}[]
\KeywordTok{ggplot}\NormalTok{(NHice, }\KeywordTok{aes}\NormalTok{(Month_Name, Extent, }\DataTypeTok{colour =} \KeywordTok{factor}\NormalTok{(season))) }\OperatorTok{+}
\StringTok{  }\KeywordTok{geom_point}\NormalTok{() }\OperatorTok{+}\StringTok{ }\KeywordTok{facet_wrap}\NormalTok{(}\OperatorTok{~}\StringTok{ }\NormalTok{season, }\DataTypeTok{ncol =} \DecValTok{4}\NormalTok{, }\DataTypeTok{scales =} \StringTok{"free"}\NormalTok{) }
\end{Highlighting}
\end{Shaded}

\includegraphics{HW_03_Judell_Halfpenny_Andrew_2018_files/figure-latex/unnamed-chunk-11-1.pdf}

\begin{Shaded}
\begin{Highlighting}[]
\NormalTok{bp<-}\KeywordTok{boxplot}\NormalTok{(Extent }\OperatorTok{~}\StringTok{ }\NormalTok{Year, }\DataTypeTok{data =}\NormalTok{ NHice, }\DataTypeTok{col =}\NormalTok{ NHice}\OperatorTok{$}\NormalTok{Month,}\DataTypeTok{main=}\StringTok{"Variability of Sea Ice Extent by Month"}\NormalTok{)}
\end{Highlighting}
\end{Shaded}

\includegraphics{HW_03_Judell_Halfpenny_Andrew_2018_files/figure-latex/unnamed-chunk-11-2.pdf}

\subsubsection{2.6) Last, make a line plot of sea ice by month with
different lines as different years. Gussy it up with colors by year, a
different theme, and whatever other annotations, changes to axes, etc.,
you think best show the story of this data. For ideas, see the
lab.}\label{last-make-a-line-plot-of-sea-ice-by-month-with-different-lines-as-different-years.-gussy-it-up-with-colors-by-year-a-different-theme-and-whatever-other-annotations-changes-to-axes-etc.-you-think-best-show-the-story-of-this-data.-for-ideas-see-the-lab.}

\begin{Shaded}
\begin{Highlighting}[]
\KeywordTok{ggplot}\NormalTok{(}\DataTypeTok{data =}\NormalTok{ NHice,}\DataTypeTok{mapping =} \KeywordTok{aes}\NormalTok{(}\DataTypeTok{x=}\NormalTok{ Month_Name, }\DataTypeTok{y =}\NormalTok{ Extent, }\DataTypeTok{group =}\NormalTok{ Year, }\DataTypeTok{col =}\NormalTok{ Year)) }\OperatorTok{+}\StringTok{ }\KeywordTok{geom_line}\NormalTok{()}
\end{Highlighting}
\end{Shaded}

\includegraphics{HW_03_Judell_Halfpenny_Andrew_2018_files/figure-latex/unnamed-chunk-12-1.pdf}


\end{document}
