\documentclass[ignorenonframetext,]{beamer}
\setbeamertemplate{caption}[numbered]
\setbeamertemplate{caption label separator}{: }
\setbeamercolor{caption name}{fg=normal text.fg}
\beamertemplatenavigationsymbolsempty
\usepackage{lmodern}
\usepackage{amssymb,amsmath}
\usepackage{ifxetex,ifluatex}
\usepackage{fixltx2e} % provides \textsubscript
\ifnum 0\ifxetex 1\fi\ifluatex 1\fi=0 % if pdftex
  \usepackage[T1]{fontenc}
  \usepackage[utf8]{inputenc}
\else % if luatex or xelatex
  \ifxetex
    \usepackage{mathspec}
  \else
    \usepackage{fontspec}
  \fi
  \defaultfontfeatures{Ligatures=TeX,Scale=MatchLowercase}
\fi
\usetheme[]{CambridgeUS}
\usecolortheme{wolverine}
\usefonttheme{serif}
% use upquote if available, for straight quotes in verbatim environments
\IfFileExists{upquote.sty}{\usepackage{upquote}}{}
% use microtype if available
\IfFileExists{microtype.sty}{%
\usepackage{microtype}
\UseMicrotypeSet[protrusion]{basicmath} % disable protrusion for tt fonts
}{}
\newif\ifbibliography
\hypersetup{
            pdftitle={BIOL-607 Biological Data Analysis Homework 4: Hypothesis Testing},
            pdfauthor={Andrew Judell-Halfpenny},
            pdfborder={0 0 0},
            breaklinks=true}
\urlstyle{same}  % don't use monospace font for urls
\usepackage{color}
\usepackage{fancyvrb}
\newcommand{\VerbBar}{|}
\newcommand{\VERB}{\Verb[commandchars=\\\{\}]}
\DefineVerbatimEnvironment{Highlighting}{Verbatim}{commandchars=\\\{\}}
% Add ',fontsize=\small' for more characters per line
\usepackage{framed}
\definecolor{shadecolor}{RGB}{248,248,248}
\newenvironment{Shaded}{\begin{snugshade}}{\end{snugshade}}
\newcommand{\KeywordTok}[1]{\textcolor[rgb]{0.13,0.29,0.53}{\textbf{#1}}}
\newcommand{\DataTypeTok}[1]{\textcolor[rgb]{0.13,0.29,0.53}{#1}}
\newcommand{\DecValTok}[1]{\textcolor[rgb]{0.00,0.00,0.81}{#1}}
\newcommand{\BaseNTok}[1]{\textcolor[rgb]{0.00,0.00,0.81}{#1}}
\newcommand{\FloatTok}[1]{\textcolor[rgb]{0.00,0.00,0.81}{#1}}
\newcommand{\ConstantTok}[1]{\textcolor[rgb]{0.00,0.00,0.00}{#1}}
\newcommand{\CharTok}[1]{\textcolor[rgb]{0.31,0.60,0.02}{#1}}
\newcommand{\SpecialCharTok}[1]{\textcolor[rgb]{0.00,0.00,0.00}{#1}}
\newcommand{\StringTok}[1]{\textcolor[rgb]{0.31,0.60,0.02}{#1}}
\newcommand{\VerbatimStringTok}[1]{\textcolor[rgb]{0.31,0.60,0.02}{#1}}
\newcommand{\SpecialStringTok}[1]{\textcolor[rgb]{0.31,0.60,0.02}{#1}}
\newcommand{\ImportTok}[1]{#1}
\newcommand{\CommentTok}[1]{\textcolor[rgb]{0.56,0.35,0.01}{\textit{#1}}}
\newcommand{\DocumentationTok}[1]{\textcolor[rgb]{0.56,0.35,0.01}{\textbf{\textit{#1}}}}
\newcommand{\AnnotationTok}[1]{\textcolor[rgb]{0.56,0.35,0.01}{\textbf{\textit{#1}}}}
\newcommand{\CommentVarTok}[1]{\textcolor[rgb]{0.56,0.35,0.01}{\textbf{\textit{#1}}}}
\newcommand{\OtherTok}[1]{\textcolor[rgb]{0.56,0.35,0.01}{#1}}
\newcommand{\FunctionTok}[1]{\textcolor[rgb]{0.00,0.00,0.00}{#1}}
\newcommand{\VariableTok}[1]{\textcolor[rgb]{0.00,0.00,0.00}{#1}}
\newcommand{\ControlFlowTok}[1]{\textcolor[rgb]{0.13,0.29,0.53}{\textbf{#1}}}
\newcommand{\OperatorTok}[1]{\textcolor[rgb]{0.81,0.36,0.00}{\textbf{#1}}}
\newcommand{\BuiltInTok}[1]{#1}
\newcommand{\ExtensionTok}[1]{#1}
\newcommand{\PreprocessorTok}[1]{\textcolor[rgb]{0.56,0.35,0.01}{\textit{#1}}}
\newcommand{\AttributeTok}[1]{\textcolor[rgb]{0.77,0.63,0.00}{#1}}
\newcommand{\RegionMarkerTok}[1]{#1}
\newcommand{\InformationTok}[1]{\textcolor[rgb]{0.56,0.35,0.01}{\textbf{\textit{#1}}}}
\newcommand{\WarningTok}[1]{\textcolor[rgb]{0.56,0.35,0.01}{\textbf{\textit{#1}}}}
\newcommand{\AlertTok}[1]{\textcolor[rgb]{0.94,0.16,0.16}{#1}}
\newcommand{\ErrorTok}[1]{\textcolor[rgb]{0.64,0.00,0.00}{\textbf{#1}}}
\newcommand{\NormalTok}[1]{#1}

% Prevent slide breaks in the middle of a paragraph:
\widowpenalties 1 10000
\raggedbottom

\AtBeginPart{
  \let\insertpartnumber\relax
  \let\partname\relax
  \frame{\partpage}
}
\AtBeginSection{
  \ifbibliography
  \else
    \let\insertsectionnumber\relax
    \let\sectionname\relax
    \frame{\sectionpage}
  \fi
}
\AtBeginSubsection{
  \let\insertsubsectionnumber\relax
  \let\subsectionname\relax
  \frame{\subsectionpage}
}

\setlength{\parindent}{0pt}
\setlength{\parskip}{6pt plus 2pt minus 1pt}
\setlength{\emergencystretch}{3em}  % prevent overfull lines
\providecommand{\tightlist}{%
  \setlength{\itemsep}{0pt}\setlength{\parskip}{0pt}}
\setcounter{secnumdepth}{0}

\title{BIOL-607 Biological Data Analysis Homework 4: Hypothesis Testing}
\author{Andrew Judell-Halfpenny}
\date{October 6, 2018}

\begin{document}
\frame{\titlepage}

\begin{frame}[fragile]

\begin{Shaded}
\begin{Highlighting}[]
\CommentTok{#1. load libraries}
\KeywordTok{library}\NormalTok{(devtools);}\KeywordTok{library}\NormalTok{(revealjs);}\KeywordTok{library}\NormalTok{(ggvis);}\KeywordTok{library}\NormalTok{(phangorn)}
\KeywordTok{library}\NormalTok{(ggtree);}\KeywordTok{library}\NormalTok{(treeio);}\KeywordTok{library}\NormalTok{(gganimate);}\KeywordTok{library}\NormalTok{(forcats)}
\KeywordTok{library}\NormalTok{(tidyr);}\KeywordTok{library}\NormalTok{(tidygraph);}\KeywordTok{library}\NormalTok{(lubridate);}\KeywordTok{library}\NormalTok{(ggplot2)}
\KeywordTok{library}\NormalTok{(animation);}\KeywordTok{library}\NormalTok{(readr);}\KeywordTok{library}\NormalTok{(abd);}\KeywordTok{library}\NormalTok{(dplyr);}\KeywordTok{library}\NormalTok{(magrittr)}
\end{Highlighting}
\end{Shaded}

\end{frame}

\begin{frame}{Part 1-W\&S Chapter 6 questions 15, 21, 29}

\end{frame}

\begin{frame}{15) For the following alternative hypotheses, give the
appropriate null hypothesis.}

\begin{block}{a) Ha = Pygmy mammoths and continental mammoths differ in
their mean femur lengths.}

\begin{quote}
Ho = Mean femur lengths for Pygmy mammoths and continental mammoths
equal.
\end{quote}

\end{block}

\begin{block}{b) Ha = Patients who take phentermine and topira-mate lose
weight at a different rate than control patients without these drugs.}

\begin{quote}
Ho = Patients who take phentermine and topira-mate lose weight at a the
same rate as control patients without these drugs.
\end{quote}

\end{block}

\begin{block}{c) Ha = Patients who take phentermine and topiramate have
different proportions of their babies born with cleft palates than do
patients not taking these drugs.}

\begin{quote}
Ho = Patients who take phentermine and topiramate the same proportions
of their babies born with cleft palates than do patients not taking
these drugs.
\end{quote}

\end{block}

\begin{block}{d) Ha = Shoppers on average buy different amounts of candy
when Christmas music is playing in the shop compared to when the usual
type of music is playing.}

\begin{quote}
Ho =Shoppers on average buy the same amount of candy when Christmas
music is playing in the shop compared to when the usual type of music is
playin
\end{quote}

\end{block}

\begin{block}{e) Ha = Male white-collared manakins (a tropical bird)
dance more often when females are present than when they are absent.}

\begin{quote}
Ho = Male white-collared manakins (a tropical bird) dance exactly as
often when females are present as when they are absent.
\end{quote}

\end{block}

\begin{block}{21) Imagine that two researchers independently carry out
clinical trials to test the same null hypothesis, that COX-2 selective
inhibitors (which are used to treat arthritis) have no effect on the
risk of cardiac arrest. They use the same population for their study,
but one experimenter uses a sample size of 60 participants, whereas the
other uses a sample size of 100. Assume that all other aspects of the
studies, including significance levels, are the same between the two
studies.}

\end{block}

\end{frame}

\begin{frame}

\begin{block}{21.a) Which study has the higher probability of a Type II
error, the 60-participant study or the 100-participant study?}

\begin{quote}
The 60-participant study has a greater probability of producing a Type
II error than the 100-participant study. This is proportiion of
participants chosen from random sample from the 60 participant study
that produce a result that leads to rejection of a false null hypothesis
is greater than the proportion in the 100 participant study.
\end{quote}

\end{block}

\begin{block}{b) Which study has higher power?}

\begin{quote}
The 100-participant study has greater power because of its reduced
likelihood of a Type II error.
\end{quote}

\end{block}

\begin{block}{c) Which study has the higher probability of a Type I
error?}

\begin{quote}
The study with 100 patients has a higher probability of a Typer I error
(false positive)
\end{quote}

\end{block}

\begin{block}{d) Should the tests be one-tailed or two-tailed? Explain.}

\begin{quote}
In each one of these statistical tests the null hypothesis is that there
is no difference (up or down) between the two populations and therefor
requires a two-tailed test.
\end{quote}

\end{block}

\end{frame}

\begin{frame}[fragile]{29) A team of researchers conducted 100
independent hypothesis tests using a significance level of α = 0.05.}

\begin{block}{a) If all 100 null hypotheses were true, what is the
probability that the researchers would reject none of them?}

\begin{quote}
At an alpha level of 0.05, the probability of 100 out of 100 true null
hypotheses can be calculated from the discrete binomial distribution
function with success defined as a type I error, and a rejection of the
null hypothesis. 0.59\%, (less than 1\&)
\end{quote}

\begin{Shaded}
\begin{Highlighting}[]
\NormalTok{pbnn<-}\KeywordTok{dbinom}\NormalTok{(}\DecValTok{0}\NormalTok{, }\DecValTok{100}\NormalTok{, }\FloatTok{0.05}\NormalTok{)}
\NormalTok{pbnmp<-pbnn }\OperatorTok{*}\StringTok{ }\DecValTok{100}
\KeywordTok{paste0}\NormalTok{(pbnmp,}\DataTypeTok{sep=}\StringTok{"%"}\NormalTok{)}
\end{Highlighting}
\end{Shaded}

\begin{verbatim}
## [1] "0.592052922033403%"
\end{verbatim}

\end{block}

\begin{block}{b) If all 100 null hypotheses were true, how many of these
tests on average are expected to reject the null hypothesis?}

\begin{quote}
Give 100 true null hypotheses, and the previously stated alpha level
(0.05), 5 tests are expected to produce results that suggest that the
researchers should reject the null hypothesis.
\end{quote}

\end{block}

\end{frame}

\begin{frame}{Part 2 W \& S Chapter 7 questions 22}

\end{frame}

\begin{frame}[fragile]{22) In a test of Murphy's law, pieces of toast
were buttered on one side and then dropped. Murphy's law predicts that
they will land butter-side down. Out of 9821 slices of toast dropped,
6101 landed butter-side down. (Believe it or not, these are real
data!10) \textbar{} a) What is a 95\% confidence interval for the
probability of a piece of toast landing butter-side down?}

\begin{quote}
manually calculating the 95\% confidence interval for the proportion of
toast slices that fall butter side down.
\end{quote}

\begin{Shaded}
\begin{Highlighting}[]
\NormalTok{n=}\DecValTok{9821}
\NormalTok{p.hat=}\DecValTok{6101}\OperatorTok{/}\NormalTok{n}
\NormalTok{p.hat}
\end{Highlighting}
\end{Shaded}

\begin{verbatim}
## [1] 0.6212198
\end{verbatim}

\begin{Shaded}
\begin{Highlighting}[]
\NormalTok{z=}\FloatTok{1.96}\NormalTok{  ### for 95% condifence interval}
\NormalTok{se.p.est=((p.hat}\OperatorTok{*}\NormalTok{(}\DecValTok{1}\OperatorTok{-}\NormalTok{p.hat)}\OperatorTok{/}\NormalTok{n)}\OperatorTok{^}\NormalTok{.}\DecValTok{5}\NormalTok{)}
\NormalTok{se.p<-}\ControlFlowTok{function}\NormalTok{(x)  ((p.hat}\OperatorTok{*}\NormalTok{(}\DecValTok{1}\OperatorTok{-}\NormalTok{p.hat)}\OperatorTok{/}\NormalTok{n)}\OperatorTok{^}\NormalTok{.}\DecValTok{5}\NormalTok{)}
\CommentTok{# from}
\NormalTok{p.hat }\OperatorTok{-}\StringTok{ }\NormalTok{(z}\OperatorTok{*}\NormalTok{se.p.est)}
\end{Highlighting}
\end{Shaded}

\begin{verbatim}
## [1] 0.611626
\end{verbatim}

\begin{Shaded}
\begin{Highlighting}[]
\CommentTok{# to}
\NormalTok{p.hat }\OperatorTok{+}\StringTok{ }\NormalTok{(z}\OperatorTok{*}\NormalTok{se.p.est)}
\end{Highlighting}
\end{Shaded}

\begin{verbatim}
## [1] 0.6308137
\end{verbatim}

\begin{quote}
the binom.test function performs an automated version of the previously
run code
\end{quote}

\begin{Shaded}
\begin{Highlighting}[]
\NormalTok{dropped.bread.proportion<-}\KeywordTok{binom.test}\NormalTok{(}\DecValTok{6101}\NormalTok{, }\DataTypeTok{n =} \DecValTok{9821}\NormalTok{)}
\NormalTok{dropped.bread.proportion}
\end{Highlighting}
\end{Shaded}

\begin{verbatim}
## 
## 
## 
## data:  6101 out of 9821
## number of successes = 6101, number of trials = 9821, p-value <
## 2.2e-16
## alternative hypothesis: true probability of success is not equal to 0.5
## 95 percent confidence interval:
##  0.6115404 0.6308273
## sample estimates:
## probability of success 
##              0.6212198
\end{verbatim}

\begin{block}{b) Using the results of part (a), is it plausible that
there is a 50:50 chance of the toast landing butter-side down or
butter-side up?}

\begin{quote}
Given that the 95\% confidence interval for the proportion of the
population of butter side down slices of dropped bread is from 0.611626
to 0.6301837 and doesn't include 0.5, it would be incorrect to
characterize the chances as 50:50. Using the available data we can
produce a p-value for for the characterization of the proportion
estimate as 50:50.
\end{quote}

\begin{Shaded}
\begin{Highlighting}[]
\CommentTok{# total number of dropped bread slices}
\NormalTok{n=}\DecValTok{9821}
\CommentTok{# proportion of dropped bread slices that land butter side down}
\NormalTok{p.hat=}\DecValTok{6101}\OperatorTok{/}\NormalTok{n}
\NormalTok{p.hat}
\end{Highlighting}
\end{Shaded}

\begin{verbatim}
## [1] 0.6212198
\end{verbatim}

\begin{Shaded}
\begin{Highlighting}[]
\CommentTok{# assuming a 50:50 chance, the nnumber of butter side down slices in 9821 trials}
\NormalTok{s.not<-n}\OperatorTok{/}\DecValTok{2}
\KeywordTok{round}\NormalTok{(s.not)}
\end{Highlighting}
\end{Shaded}

\begin{verbatim}
## [1] 4910
\end{verbatim}

\begin{Shaded}
\begin{Highlighting}[]
\CommentTok{# with proportion 50:50 (obviously)}
\NormalTok{p.not<-s.not}\OperatorTok{/}\NormalTok{n}
\NormalTok{p.not}
\end{Highlighting}
\end{Shaded}

\begin{verbatim}
## [1] 0.5
\end{verbatim}

\begin{Shaded}
\begin{Highlighting}[]
\NormalTok{fifty.fifty.dropped.bread.prop<-}\KeywordTok{binom.test}\NormalTok{(}\KeywordTok{round}\NormalTok{(s.not), }\DataTypeTok{n =} \DecValTok{9821}\NormalTok{)}
\NormalTok{fifty.fifty.dropped.bread.prop}
\end{Highlighting}
\end{Shaded}

\begin{verbatim}
## 
## 
## 
## data:  round(s.not) out of 9821
## number of successes = 4910, number of trials = 9821, p-value = 1
## alternative hypothesis: true probability of success is not equal to 0.5
## 95 percent confidence interval:
##  0.4900107 0.5098875
## sample estimates:
## probability of success 
##              0.4999491
\end{verbatim}

\begin{quote}
To produce a p-value for the hypothesis that the proportion of
butter-down bread slices can be characterized as 50:50 despite a priori
estimate of \textasciitilde{}0.6116 via sampling, the prop.test function
can be used with the expected number of butter-down slices for 9821
trials given probability of 50:50
\end{quote}

\begin{Shaded}
\begin{Highlighting}[]
\KeywordTok{prop.test}\NormalTok{(}\KeywordTok{round}\NormalTok{(s.not), }\DataTypeTok{n=}\NormalTok{n,}\DataTypeTok{p=}\NormalTok{p.hat,}\DataTypeTok{correct=}\NormalTok{F)        }
\end{Highlighting}
\end{Shaded}

\begin{verbatim}
## 
##  1-sample proportions test without continuity correction
## 
## data:  round(s.not) out of n
## X-squared = 613.81, df = 1, p-value < 2.2e-16
## alternative hypothesis: true p is not equal to 0.6212198
## 95 percent confidence interval:
##  0.4900623 0.5098359
## sample estimates:
##         p 
## 0.4999491
\end{verbatim}

\end{block}

\end{frame}

\begin{frame}{3) From the Lab: A Simulation of Many SDs and Alphas
\textbar{} Here's the exercise we started in lab. Feel free to look back
copiously at the lab handout if you're getting stuck. Remember, for each
step, write-out in comments what you want to do, and then follow behing
with code. \textbar{} Now, let's assume an average population-wide
resting heart rate of 80 beats per minute with a standard deviation of 6
BPM. \textbar{} A given drug speeds people's heart rates up on average
by 5 BPM. What sample size do we need to achieve a power of 0.8?}

\end{frame}

\begin{frame}{3.1) Start the Simulation}

\begin{quote}
Make a simulated data frame to look at the effects of multiple sample
sizes: from 1-20, with 500 simulations per sample size, and also
multiple SD values, from 3 through 10 (just 3:10, no need for
non-integer values). Using the crossing function with your intitial data
frame of just sample sizes and a vector of sd values to start generate
samples from the appropriate random normal distribution.
\end{quote}

\end{frame}

\begin{frame}{3.2) Z!}

\begin{quote}
OK, now that you've done that, calculate the results from z-tests. Plot
p by sample size, using facet\_wrap for different SD values.
\end{quote}

\end{frame}

\begin{frame}{3.3) P and Power}

\begin{quote}
Now plot power for an alpha of 0.05, but use color for different SD
values. Include our threshold power of 0.8.
\end{quote}

\end{frame}

\begin{frame}{3.4) Many alphas}

\begin{quote}
Last, use crossing again to explore changing alphas from 0.01 to 0.1.
Plot power curves with different alphas as different colors, and use
faceting to look at different SDs.
\end{quote}

\end{frame}

\begin{frame}{3.5) What does it all mean? What do you learn about how
alpha and SD affect power?}

\end{frame}

\begin{frame}{3.6) How do you think that changing the effect size would
affect power?}

\end{frame}

\begin{frame}{You can just answer this without coding out anything.
Based on what we've learned so far - what do you think?}

\end{frame}

\end{document}
